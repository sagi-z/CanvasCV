
\begin{DoxyItemize}
\item \hyperlink{tutscreentext}{Leaving the console} (start here)
\item \hyperlink{tutmsgbox}{Using the Msg\+Box}
\item \hyperlink{tutshapes}{Creating shapes}
\item \hyperlink{tutpersistency}{Persisting shapes}
\item \hyperlink{tutbuttons}{Using buttons}
\item \hyperlink{tutcheckbox}{Using Check\+Boxes (and Radio\+Buttons and a Selection\+Box)}
\item \hyperlink{tutlayouts}{Auto widget layouts} (puting widgets together)
\item \href{https://sagi-z.github.io/CanvasCV/support}{\tt more tutorials and help}
\end{DoxyItemize}

(see also the \href{examples.html}{\tt examples}) \hypertarget{tutscreentext}{}\section{Leaving the console}\label{tutscreentext}
Open\+CV examples and tutorials make heavy use of the console to commuincate with the user.

To start using the G\+UI instead of the console the first thing you need is to get to know the \hyperlink{classcanvascv_1_1Canvas}{canvascv\+::\+Canvas} class.\hypertarget{tutscreentext_screentext_s1}{}\subsection{Introducing the Canvas\+C\+V class}\label{tutscreentext_screentext_s1}
The Canvas class is associated with an Open\+CV window and it gives you another virtual layer on top of your displayed Mat.

This canvas class encapsulates {\bfseries a lot} of work for you, so you can focus on the CV. It also handles key presses and mouse events.

The basic flow is\+:
\begin{DoxyItemize}
\item Define a Canvas for a window of a specific size, usually to match the image you\textquotesingle{}re displayng. 
\begin{DoxyCode}
Canvas c(\textcolor{stringliteral}{"winName"}, frame.size());
\end{DoxyCode}

\item Create an Open\+CV window without resizing so the widgets cannot be stretched 
\begin{DoxyCode}
namedWindow(\textcolor{stringliteral}{"winName"}, WINDOW\_AUTOSIZE); \textcolor{comment}{// disable mouse resize}
\end{DoxyCode}

\item Let the Canvas\+CV handle your mouse events 
\begin{DoxyCode}
c.\hyperlink{classcanvascv_1_1Canvas_acf6e5d4b40aec610b0dc8c4f6bf93ac1}{setMouseCallback}(); \textcolor{comment}{// optional for mouse usage (see also example\_selectbox.cpp)}
\end{DoxyCode}

\item Create your widget on the Canvas (or in a frame/layout) for example\+: 
\begin{DoxyCode}
\textcolor{keyword}{auto} msgBox = MsgBox::create(c, \textcolor{stringliteral}{"Do you really want to do that?"}, \{\textcolor{stringliteral}{"Yes"}, \textcolor{stringliteral}{"No"}\});
\end{DoxyCode}

\item Redraw on your frame/image in your regular Open\+CV loop, as in\+: 
\begin{DoxyCode}
\textcolor{keywordtype}{int} delay = 1000/25; \textcolor{comment}{// for a single image, delay should be 0}
\textcolor{keywordtype}{int} key = -1;
Mat out;
\textcolor{keywordflow}{while} (\textcolor{keyword}{true})
\{
    c.\hyperlink{classcanvascv_1_1Canvas_a018c66e277de7904b8146ea3f3feebdd}{redrawOn}(frame, out);   \textcolor{comment}{// draws your virtual GUI layer on top of the 'frame'}
    c.\hyperlink{classcanvascv_1_1Canvas_acaf9494a5668046dd0a8908aa97a7a43}{imshow}(out);            \textcolor{comment}{// you can also use imshow directly for "winName"}
    key = c.\hyperlink{classcanvascv_1_1Canvas_a59397db05f5d9e45264f626f6a2ae528}{waitKeyEx}(delay); \textcolor{comment}{// forwards key presses to widgets and shapes}
\}
\end{DoxyCode}

\end{DoxyItemize}\hypertarget{tutscreentext_screentext_s2}{}\subsection{Some real use cases}\label{tutscreentext_screentext_s2}
\hypertarget{tutscreentext_screentext_s2_1}{}\subsubsection{Exiting on fatal errors}\label{tutscreentext_screentext_s2_1}
Your application/utility might need command line arguments, or cannot continue for some reason.

A simple write to the console before exiting is usually not enough to get the user attention.

For these simple cases you have a simple shortcut in \hyperlink{classcanvascv_1_1Canvas_add93c0d5cc1e9b49f97510952a8a1961}{canvascv\+::\+Canvas\+::fatal()}\+: 
\begin{DoxyCode}
\textcolor{preprocessor}{#include <canvascv/canvas.h>}

\textcolor{keyword}{using namespace }\hyperlink{namespacecanvascv}{canvascv};

\textcolor{keywordtype}{int} main(\textcolor{keywordtype}{int} argc, \textcolor{keywordtype}{char} **argv)
\{
    --argc;
    ++argv;
    \textcolor{keywordflow}{if} (! argc)
    \{
        \hyperlink{classcanvascv_1_1Canvas_add93c0d5cc1e9b49f97510952a8a1961}{Canvas::fatal}(\textcolor{stringliteral}{"Must get a path to an image as a parameter"} , -1);
    \}
    \textcolor{keywordflow}{return} 0;
\}
\end{DoxyCode}


which, besides an output to S\+T\+D\+E\+RR, gives you\+:  ~\newline
\hypertarget{tutscreentext_screentext_s2_2}{}\subsubsection{Displaying user text in a fixed location}\label{tutscreentext_screentext_s2_2}
Instead of displaying the help message or any other fixed text to the console use the \hyperlink{classcanvascv_1_1Canvas_ae68d3277e738d349232400b38f0e5f9e}{canvascv\+::\+Canvas\+::enable\+Screen\+Text()} and canvaccv\+::\+Canvas\+::set\+Screen\+Text(string) methods.

This flow is a little closer to the regular usage of the {\itshape Canvas}, but you don\textquotesingle{}t need to create a widget of your own.

Adding to the previous example (we want to display this on an image)\+: 
\begin{DoxyCode}
\textcolor{preprocessor}{#include <canvascv/canvas.h>}

\textcolor{keyword}{using namespace }\hyperlink{namespacecanvascv}{canvascv};

\textcolor{keywordtype}{void} help(\hyperlink{classcanvascv_1_1Canvas}{Canvas} &c)
\{
    \textcolor{keyword}{static} \textcolor{keywordtype}{bool} showHelp = \textcolor{keyword}{true};
    \textcolor{keyword}{static} \textcolor{keywordtype}{string} helpMsg =
            \textcolor{stringliteral}{"Usage:\(\backslash\)n"}
            \textcolor{stringliteral}{"=====\(\backslash\)n"}
            \textcolor{stringliteral}{"h: toggle usage message\(\backslash\)n"}
            \textcolor{stringliteral}{"*: toggle canvas on/off\(\backslash\)n"}
            \textcolor{stringliteral}{"q: exit"};


    \textcolor{keywordflow}{if} (showHelp) c.\hyperlink{classcanvascv_1_1Canvas_aaedea276b82a8a4cfc0895ae81113cfd}{setScreenText}(helpMsg);
    \textcolor{keywordflow}{else} c.\hyperlink{classcanvascv_1_1Canvas_aaedea276b82a8a4cfc0895ae81113cfd}{setScreenText}(\textcolor{stringliteral}{""});

    showHelp = ! showHelp;
\}

\textcolor{keywordtype}{int} main(\textcolor{keywordtype}{int} argc, \textcolor{keywordtype}{char} **argv)
\{
    --argc;
    ++argv;
    \textcolor{keywordflow}{if} (! argc)
    \{
        \hyperlink{classcanvascv_1_1Canvas_add93c0d5cc1e9b49f97510952a8a1961}{Canvas::fatal}(\textcolor{stringliteral}{"Must get a path to an image as a parameter"} , -1);
    \}

    Mat image = imread(argv[0]);
    \textcolor{keywordflow}{if} (image.empty())
    \{
        \hyperlink{classcanvascv_1_1Canvas_add93c0d5cc1e9b49f97510952a8a1961}{Canvas::fatal}(\textcolor{keywordtype}{string}(\textcolor{stringliteral}{"Cannot load image "}) + argv[0], -2);
    \}

    \hyperlink{classcanvascv_1_1Canvas}{Canvas} c(\textcolor{stringliteral}{"Canvas"}, image.size());
    c.\hyperlink{classcanvascv_1_1Canvas_ae68d3277e738d349232400b38f0e5f9e}{enableScreenText}(); \textcolor{comment}{// see it's documentation}

    help(c);

    namedWindow(\textcolor{stringliteral}{"Canvas"}, WINDOW\_AUTOSIZE); \textcolor{comment}{// disable mouse resize}

    \textcolor{keywordtype}{int} key = 0;
    Mat out;
    \textcolor{keywordflow}{do}
    \{
        \textcolor{keywordflow}{switch} (key)
        \{
        \textcolor{keywordflow}{case} \textcolor{charliteral}{'h'}:
            help(c);
            \textcolor{keywordflow}{break};
        \textcolor{keywordflow}{case} \textcolor{charliteral}{'*'}:
            c.\hyperlink{classcanvascv_1_1Canvas_aba149ea25c6cdad2673133a060355954}{setOn}(! c.\hyperlink{classcanvascv_1_1Canvas_afe6a2955a5bbee8903350b4fba3f4473}{getOn}());
            \textcolor{keywordflow}{break};
        \}

        c.\hyperlink{classcanvascv_1_1Canvas_a018c66e277de7904b8146ea3f3feebdd}{redrawOn}(image, out);   \textcolor{comment}{// draw the canvas on the image copy}

        imshow(\textcolor{stringliteral}{"Canvas"}, out);    \textcolor{comment}{// using cv::imshow works fine}

        key = c.\hyperlink{classcanvascv_1_1Canvas_a59397db05f5d9e45264f626f6a2ae528}{waitKeyEx}(); \textcolor{comment}{// GUI and callbacks happen here}

    \} \textcolor{keywordflow}{while} (key != \textcolor{charliteral}{'q'});

    destroyAllWindows();

    \textcolor{keywordflow}{return} 0;
\}
\end{DoxyCode}


Some notes here\+:
\begin{DoxyItemize}
\item Note the continued use of \hyperlink{classcanvascv_1_1Canvas_add93c0d5cc1e9b49f97510952a8a1961}{canvascv\+::\+Canvas\+::fatal()}.
\item Note that we didn\textquotesingle{}t need to use \hyperlink{classcanvascv_1_1Canvas_acf6e5d4b40aec610b0dc8c4f6bf93ac1}{canvascv\+::\+Canvas\+::set\+Mouse\+Callback()} or create a widget of our own here.
\item When you \hyperlink{classcanvascv_1_1Canvas_ae68d3277e738d349232400b38f0e5f9e}{canvascv\+::\+Canvas\+::enable\+Screen\+Text()} you can configure it\textquotesingle{}s display.
\item As you can see the {\itshape Canvas} instance can be completely turned on and off with the \hyperlink{classcanvascv_1_1Canvas_aba149ea25c6cdad2673133a060355954}{canvascv\+::\+Canvas\+::set\+On()} method.
\item \hyperlink{classcanvascv_1_1Canvas_a59397db05f5d9e45264f626f6a2ae528}{canvascv\+::\+Canvas\+::wait\+Key\+Ex()} knows to update it\textquotesingle{}s internal shapes and widgets even if you pass 0 as a blocking delay indicator.
\item When executed with a path to an image, this gives you (depends on your image)\+:  ~\newline

\end{DoxyItemize}\hypertarget{tutscreentext_screentext_s2_3}{}\subsubsection{Displaying text where ever you want}\label{tutscreentext_screentext_s2_3}
To display text where you want you need the \hyperlink{classcanvascv_1_1Text}{canvascv\+::\+Text} widget.

You can create position it on the Canvas at any XY location.

Add these lines to the previous example, after creating the named\+Window()\+: 
\begin{DoxyCode}
\textcolor{keyword}{using namespace }\hyperlink{namespacecanvascv}{canvascv};
    \textcolor{comment}{//...}

    \textcolor{keyword}{auto} txt = \hyperlink{classcanvascv_1_1Text_a7f3552263b6f185f78d90549e7ac38f7}{Text::create}(c, \textcolor{stringliteral}{"Target Acquired!"});
    txt->setFontHeight(50);
    txt->setOutlineColor(\hyperlink{classcanvascv_1_1Colors_a10aff24c53edf45b038d0636b061f9c2}{Colors::Red});
    txt->setThickness(2);
    txt->setLocation(\{30, image.rows / 2 - 15\});

    \textcolor{comment}{//...}
\end{DoxyCode}


Notes\+:
\begin{DoxyItemize}
\item All widgets have a static create methods, which is the only way to create them.
\item The \hyperlink{classcanvascv_1_1Text_a7f3552263b6f185f78d90549e7ac38f7}{canvascv\+::\+Text\+::create()} will return a shared\+\_\+ptr$<$\+Text$>$ instance, which you don\textquotesingle{}t have to keep since another one is kept by the \hyperlink{classcanvascv_1_1Layout}{canvascv\+::\+Layout}.
\item There are many Colors constants to choose from, see \hyperlink{classcanvascv_1_1Colors}{canvascv\+::\+Colors}.
\item The \hyperlink{classcanvascv_1_1Widget_a8a36b15a1c777baffbb4fcd4ccda3c45}{canvascv\+::\+Widget\+::set\+Location()} gives specific XY postion, but a \hyperlink{classcanvascv_1_1HorizontalLayout}{canvascv\+::\+Horizontal\+Layout} could help us put this in the {\itshape C\+E\+N\+T\+ER}.
\item When executed with a path to an image, this gives you (depends on your image)\+:  ~\newline

\end{DoxyItemize}

{\bfseries That\textquotesingle{}s all for this tutorial} \hypertarget{tutmsgbox}{}\section{Using the Msg\+Box}\label{tutmsgbox}
The \hyperlink{classcanvascv_1_1MsgBox}{canvascv\+::\+Msg\+Box} is a little different from other widgets, because it is very high level.

This makes it easy to use and a good starting example for widgets that get mouse input.

The {\itshape Msg\+Box} is a \char`\"{}one shot user pressed and widget died\char`\"{} kind of widget.\hypertarget{tutmsgbox_msgbox_s1}{}\subsection{Remember the main loop}\label{tutmsgbox_msgbox_s1}
Just to remind you from the previous tutotial, the main loop looks something like this\+: 
\begin{DoxyCode}
\textcolor{preprocessor}{#include <canvascv/canvas.h>}
\textcolor{preprocessor}{#include <canvascv/widgets/msgbox.h>}

\textcolor{keyword}{using namespace }\hyperlink{namespacecanvascv}{canvascv};

\textcolor{keywordtype}{int} main(\textcolor{keywordtype}{int} argc, \textcolor{keywordtype}{char} **argv)
\{
    --argc;
    ++argv;
    \textcolor{keywordflow}{if} (! argc)
    \{
        \hyperlink{classcanvascv_1_1Canvas_add93c0d5cc1e9b49f97510952a8a1961}{Canvas::fatal}(\textcolor{stringliteral}{"Must get a path to an image as a parameter"} , -1);
    \}

    Mat image = imread(argv[0]);
    \textcolor{keywordflow}{if} (image.empty())
    \{
        \hyperlink{classcanvascv_1_1Canvas_add93c0d5cc1e9b49f97510952a8a1961}{Canvas::fatal}(\textcolor{keywordtype}{string}(\textcolor{stringliteral}{"Cannot load image "}) + argv[0], -2);
    \}

    \hyperlink{classcanvascv_1_1Canvas}{Canvas} c(\textcolor{stringliteral}{"MsgBox example"}, image.size());
    c.\hyperlink{classcanvascv_1_1Canvas_ae68d3277e738d349232400b38f0e5f9e}{enableScreenText}();

    namedWindow(\textcolor{stringliteral}{"MsgBox example"}, WINDOW\_AUTOSIZE);

    \textcolor{keywordtype}{int} key = 0;
    Mat out;
    \textcolor{keywordflow}{do}
    \{
        c.\hyperlink{classcanvascv_1_1Canvas_a018c66e277de7904b8146ea3f3feebdd}{redrawOn}(image, out);  \textcolor{comment}{// draw the canvas on the image copy}

        imshow(\textcolor{stringliteral}{"MsgBox example"}, out);

        key = c.\hyperlink{classcanvascv_1_1Canvas_a59397db05f5d9e45264f626f6a2ae528}{waitKeyEx}(); \textcolor{comment}{// GUI and callbacks happen here}
    \} \textcolor{keywordflow}{while} (key != \textcolor{charliteral}{'q'});

    destroyAllWindows();

    \textcolor{keywordflow}{return} 0;
\}
\end{DoxyCode}


We want to work on top of an existing image so you can see the transparency.

Let\textquotesingle{}s add the real code now.\hypertarget{tutmsgbox_msgbox_s2}{}\subsection{A simple modal Msg\+Box}\label{tutmsgbox_msgbox_s2}
Since a {\itshape Msg\+Box} usually requires immediate user attention, you have a way to block other G\+UI while waiting for a user reaction.

Remeber that the {\itshape Msg\+Box} is a \char`\"{}one shot user pressed and widget died\char`\"{} kind of widget. Here we\textquotesingle{}re recreating the {\itshape Msg\+Box} each time it dies.

By default a {\itshape Msg\+Box} will be placed at the center of the screen.

You can poll it at each loop with \hyperlink{classcanvascv_1_1MsgBox_a8762fe664f293389a1b823c75dc545e1}{canvascv\+::\+Msg\+Box\+::get\+User\+Selection()} or you can use it in a modal way, in which you block everything else and wait on that line of code for the user selection.

This is what we\textquotesingle{}ll do here, with this code\+: 
\begin{DoxyCode}
\textcolor{preprocessor}{#include <canvascv/canvas.h>}
\textcolor{preprocessor}{#include <canvascv/widgets/msgbox.h>}

\textcolor{keyword}{using namespace }\hyperlink{namespacecanvascv}{canvascv};

\textcolor{keywordtype}{int} main(\textcolor{keywordtype}{int} argc, \textcolor{keywordtype}{char} **argv)
\{
    --argc;
    ++argv;
    \textcolor{keywordflow}{if} (! argc)
    \{
        \hyperlink{classcanvascv_1_1Canvas_add93c0d5cc1e9b49f97510952a8a1961}{Canvas::fatal}(\textcolor{stringliteral}{"Must get a path to an image as a parameter"} , -1);
    \}

    Mat image = imread(argv[0]);
    \textcolor{keywordflow}{if} (image.empty())
    \{
        \hyperlink{classcanvascv_1_1Canvas_add93c0d5cc1e9b49f97510952a8a1961}{Canvas::fatal}(\textcolor{keywordtype}{string}(\textcolor{stringliteral}{"Cannot load image "}) + argv[0], -2);
    \}

    \hyperlink{classcanvascv_1_1Canvas}{Canvas} c(\textcolor{stringliteral}{"MsgBox example"}, image.size());
    c.\hyperlink{classcanvascv_1_1Canvas_ae68d3277e738d349232400b38f0e5f9e}{enableScreenText}();

    namedWindow(\textcolor{stringliteral}{"MsgBox example"}, WINDOW\_AUTOSIZE);

    c.\hyperlink{classcanvascv_1_1Canvas_acf6e5d4b40aec610b0dc8c4f6bf93ac1}{setMouseCallback}(); \textcolor{comment}{// optional for mouse usage (see also example\_selectbox.cpp)}

    Mat out;
    \textcolor{keywordtype}{int} userSelection = 0;
    \textcolor{keywordtype}{int} cnt = 0;
    \textcolor{keywordflow}{do}
    \{
        c.\hyperlink{classcanvascv_1_1Canvas_a018c66e277de7904b8146ea3f3feebdd}{redrawOn}(image, out);  \textcolor{comment}{// draw the canvas on the image copy}

        imshow(\textcolor{stringliteral}{"MsgBox example"}, out);

        \textcolor{comment}{// the blocking API handles GUI internally}
        userSelection = \hyperlink{classcanvascv_1_1MsgBox_a3bf0019e83e367e415da29286db2c5d0}{MsgBox::create}(c, \textcolor{keywordtype}{string}(\textcolor{stringliteral}{"Notice #"}) + to\_string(++cnt) + \textcolor{stringliteral}{" this msg"}
      , \{
                                           \textcolor{stringliteral}{"Ok"}, \textcolor{stringliteral}{"Whatever"}
                                       \})->getUserSelection(\textcolor{keyword}{true});
        c.\hyperlink{classcanvascv_1_1Canvas_aaedea276b82a8a4cfc0895ae81113cfd}{setScreenText}(CCV\_STR(\textcolor{stringliteral}{"User pressed button with index '"} << userSelection << \textcolor{stringliteral}{"'\(\backslash\)n\(\backslash\)n"}
       <<
                                \textcolor{stringliteral}{"(Choose 'Whatever' to exit)"}));
    \} \textcolor{keywordflow}{while} (userSelection == 0);

    destroyAllWindows();

    \textcolor{keywordflow}{return} 0;
\}
\end{DoxyCode}
 Notes\+:
\begin{DoxyItemize}
\item We needed \hyperlink{classcanvascv_1_1Canvas_acf6e5d4b40aec610b0dc8c4f6bf93ac1}{canvascv\+::\+Canvas\+::set\+Mouse\+Callback()} since the Canvas will intercept mouse events for the {\itshape Msg\+Box} now.
\item All widgets have a static create methods, which is the only way to create them.
\item The \hyperlink{classcanvascv_1_1MsgBox_a3bf0019e83e367e415da29286db2c5d0}{canvascv\+::\+Msg\+Box\+::create()} will return a {\itshape shared\+\_\+ptr$<$\+Msg\+Box$>$} instance, which you don\textquotesingle{}t have to keep since another one is kept by the \hyperlink{classcanvascv_1_1Layout}{canvascv\+::\+Layout}.
\item Here we\textquotesingle{}re using canvascv\+::\+Msg\+Box\+::get\+User\+Selection(true) immediatly on that {\itshape shared\+\_\+ptr}, and by passing true we\textquotesingle{}re blocked at that line of code.
\item {\itshape C\+C\+V\+\_\+\+S\+TR} lets you create a string as you would write into a stream.
\item When executed with a path to an image, this gives you (depends on your image)\+:  ~\newline

\end{DoxyItemize}\hypertarget{tutmsgbox_msgbox_s3}{}\subsection{A non modal Msg\+Box}\label{tutmsgbox_msgbox_s3}
Remeber that the Msg\+Box is a \char`\"{}one shot user pressed and widget died\char`\"{} kind of widget. Here we\textquotesingle{}re recreating the {\itshape Msg\+Box} each time it dies, but with callbacks.

If you wan to see code also using the {\itshape Msg\+Box} without callback (polling) then see \hyperlink{example_msgbox_8cpp-example}{example\+\_\+msgbox.\+cpp}.

Here we\textquotesingle{}ll be working with callbacks. The callback is called when the user presses a button.

After the callback is called the {\itshape Msg\+Box} will be destroyed automatically by the framework.

In our callback we\textquotesingle{}ll immediatly create a new {\itshape Msg\+Box} with the same callback, just changing the message.

Add these lines to the \textquotesingle{}main loop reminder code\textquotesingle{} above, after creating the {\itshape named\+Window()}\+: 
\begin{DoxyCode}
\textcolor{keyword}{using namespace }\hyperlink{namespacecanvascv}{canvascv};
    \textcolor{comment}{//...}

    c.\hyperlink{classcanvascv_1_1Canvas_acf6e5d4b40aec610b0dc8c4f6bf93ac1}{setMouseCallback}(); \textcolor{comment}{// optional for mouse usage (see also example\_selectbox.cpp)}

    \hyperlink{classcanvascv_1_1Widget_a977cbd39cf203c5866f07f3645c7e4bc}{Widget::CBUserSelection}  cb;
    \textcolor{keywordtype}{int} cnt = 0;

    \textcolor{comment}{// initialize the cb with a C++11 lambda expression.}
    \textcolor{comment}{// this will be given to the MsgBoxes we'll create.}
    cb = [&c, &cb, &cnt](Widget *w, \textcolor{keywordtype}{int} index)
    \{
        \hyperlink{classcanvascv_1_1MsgBox}{MsgBox} *pMsgBox = (\hyperlink{classcanvascv_1_1MsgBox}{MsgBox}*) w;
        c.\hyperlink{classcanvascv_1_1Canvas_aaedea276b82a8a4cfc0895ae81113cfd}{setScreenText}(CCV\_STR( \textcolor{stringliteral}{"User pressed option numer '"} << index << \textcolor{stringliteral}{"'\(\backslash\)n"} <<
                                 \textcolor{stringliteral}{"The button text is '"} << pMsgBox->\hyperlink{classcanvascv_1_1MsgBox_af97285a26857652a316387505ff7dce0}{getTextAt}(index) << \textcolor{stringliteral}{"'\(\backslash\)n\(\backslash\)n"} <<
                                 \textcolor{stringliteral}{"(press q to exit)"}));
        \hyperlink{classcanvascv_1_1MsgBox_a3bf0019e83e367e415da29286db2c5d0}{MsgBox::create}(c, CCV\_STR(\textcolor{stringliteral}{"Do you really want to do that? "} << \textcolor{stringliteral}{"("} << ++cnt << \textcolor{stringliteral}{")"}), 
      \{
                           \textcolor{stringliteral}{"Yes"}, \textcolor{stringliteral}{"No"}
                       \}, cb);
    \};

    \textcolor{comment}{// the first MsgBox the user will see is this one.}
    \textcolor{comment}{// when it ends by a mouse press, the cb will be called to create a new MsgBox.}
    \hyperlink{classcanvascv_1_1MsgBox_a3bf0019e83e367e415da29286db2c5d0}{MsgBox::create}(c, \textcolor{stringliteral}{"Do you really want to do that?"}, \{\textcolor{stringliteral}{"Yes"}, \textcolor{stringliteral}{"No"}\}, cb);

    \textcolor{comment}{//...}
\end{DoxyCode}


Notes\+:
\begin{DoxyItemize}
\item All widgets have a static {\itshape create} methods, which is the only way to create them.
\item The \hyperlink{classcanvascv_1_1MsgBox_a3bf0019e83e367e415da29286db2c5d0}{canvascv\+::\+Msg\+Box\+::create()} will return a {\itshape shared\+\_\+ptr$<$\+Msg\+Box$>$} instance, which you don\textquotesingle{}t have to keep since another one is kept by the \hyperlink{classcanvascv_1_1Layout}{canvascv\+::\+Layout}.
\item This tutorial is using C++11 lambda expressions as callbacks, but anything which has the \char`\"{}void(\+Widget$\ast$,int)\char`\"{} signature will work.
\item We needed \hyperlink{classcanvascv_1_1Canvas_acf6e5d4b40aec610b0dc8c4f6bf93ac1}{canvascv\+::\+Canvas\+::set\+Mouse\+Callback()} since the Canvas will intercept mouse events for the {\itshape Msg\+Box} now.
\item {\itshape C\+C\+V\+\_\+\+S\+TR} lets you create a string as you would write into a stream.
\item When executed with a path to an image, this gives you (depends on your image)\+:  ~\newline

\end{DoxyItemize}\hypertarget{tutmsgbox_msgbox_s4}{}\subsection{An external window modal Msg\+Box}\label{tutmsgbox_msgbox_s4}
Since a {\itshape Msg\+Box} usually requires immediate user attention, you have another way to block other G\+UI while waiting for a user reaction.

This A\+PI opens an new independent Open\+CV window, occupying just the {\itshape Msg\+Box}, and waiting for a user response.

This is a blocking A\+PI, the code will wait for a user button press.

(You actually saw something like this when we explained the \hyperlink{classcanvascv_1_1Canvas_add93c0d5cc1e9b49f97510952a8a1961}{canvascv\+::\+Canvas\+::fatal()} in a previous tutorial)


\begin{DoxyCode}
\textcolor{preprocessor}{#include <canvascv/canvas.h>}
\textcolor{preprocessor}{#include <canvascv/widgets/msgbox.h>}

\textcolor{keyword}{using namespace }\hyperlink{namespacecanvascv}{canvascv};

\textcolor{keywordtype}{int} main(\textcolor{keywordtype}{int} argc, \textcolor{keywordtype}{char} **argv)
\{
    --argc;
    ++argv;
    \textcolor{keywordflow}{if} (! argc)
    \{
        \hyperlink{classcanvascv_1_1Canvas_add93c0d5cc1e9b49f97510952a8a1961}{Canvas::fatal}(\textcolor{stringliteral}{"Must get a path to an image as a parameter"} , -1);
    \}

    Mat image = imread(argv[0]);
    \textcolor{keywordflow}{if} (image.empty())
    \{
        \hyperlink{classcanvascv_1_1Canvas_add93c0d5cc1e9b49f97510952a8a1961}{Canvas::fatal}(\textcolor{keywordtype}{string}(\textcolor{stringliteral}{"Cannot load image "}) + argv[0], -2);
    \}

    \hyperlink{classcanvascv_1_1Canvas}{Canvas} c(\textcolor{stringliteral}{"MsgBox example"}, image.size());
    c.\hyperlink{classcanvascv_1_1Canvas_ae68d3277e738d349232400b38f0e5f9e}{enableScreenText}();

    namedWindow(\textcolor{stringliteral}{"MsgBox example"}, WINDOW\_AUTOSIZE);

    c.\hyperlink{classcanvascv_1_1Canvas_acf6e5d4b40aec610b0dc8c4f6bf93ac1}{setMouseCallback}(); \textcolor{comment}{// optional for mouse usage (see also example\_selectbox.cpp)}

    Mat out;
    \textcolor{keywordtype}{int} userSelection = 0;
    \textcolor{keywordflow}{do}
    \{
        c.\hyperlink{classcanvascv_1_1Canvas_a018c66e277de7904b8146ea3f3feebdd}{redrawOn}(image, out);  \textcolor{comment}{// draw the canvas on the image copy}

        imshow(\textcolor{stringliteral}{"MsgBox example"}, out);

        \textcolor{comment}{// the blocking API handles GUI internally}
        userSelection = \hyperlink{classcanvascv_1_1MsgBox_a1eb6af15c2393ce6cda9ce277c01d200}{MsgBox::createModal}(\textcolor{stringliteral}{"Modal MsgBox"}, \textcolor{stringliteral}{"Notice this window"}, \{\textcolor{stringliteral}{"Ok"},
       \textcolor{stringliteral}{"Whatever"}\});
        c.\hyperlink{classcanvascv_1_1Canvas_aaedea276b82a8a4cfc0895ae81113cfd}{setScreenText}(CCV\_STR(\textcolor{stringliteral}{"User pressed button with index '"} << userSelection << \textcolor{stringliteral}{"'\(\backslash\)n\(\backslash\)n"}
       <<
                                \textcolor{stringliteral}{"(Choose 'Whatever' to exit)"}));
    \} \textcolor{keywordflow}{while} (userSelection == 0);

    destroyAllWindows();

    \textcolor{keywordflow}{return} 0;
\}
\end{DoxyCode}


Notes\+:
\begin{DoxyItemize}
\item Again -\/ this is a blocking A\+PI. The line which is using \hyperlink{classcanvascv_1_1MsgBox_a1eb6af15c2393ce6cda9ce277c01d200}{canvascv\+::\+Msg\+Box\+::create\+Modal()} waits for a response.
\item Currently there is no control of where the Open\+CV window will be opened.
\item {\itshape C\+C\+V\+\_\+\+S\+TR} lets you create a string as you would write into a stream.
\item When executed with a path to an image, this gives you (depends on your image)\+:  ~\newline

\end{DoxyItemize}

{\bfseries That\textquotesingle{}s all for this tutorial} \hypertarget{tutshapes}{}\section{Creating shapes}\label{tutshapes}
Canvas\+CV\textquotesingle{}s support of shapes is one of it\textquotesingle{}s strongest points.

You have 2 ways of creating shapes\+:
\begin{DoxyItemize}
\item {\bfseries From the G\+UI} (with the mouse and keyboard) -\/ for example to let the user select a polygon mask as an area of intereset for the CV.
\item {\bfseries From the code} -\/ for example creating a clickable rectangle on screen to which a {\itshape Selection\+Box} widget can be applied.
\end{DoxyItemize}\hypertarget{tutshapes_shapes_s1}{}\subsection{Creating a polygon from the G\+UI}\label{tutshapes_shapes_s1}
In this example we want to user to mark a polygon area on the image or frame.

There could be various reasons for that\+:
\begin{DoxyItemize}
\item Maybe the camera is capturing a large area and we want to operate our algorithm on a relevant dynamic subset of it.
\item The user selects a polygon area that triggers an alert when tracked objects are inside it.
\item etc.
\end{DoxyItemize}

When the user creates/modifies/deletes the polygon we\textquotesingle{}ll just print the vertices to the screen.

Here is the code\+: 
\begin{DoxyCode}
\textcolor{preprocessor}{#include <canvascv/canvas.h>}
\textcolor{preprocessor}{#include <canvascv/shapes/polygon.h>}

\textcolor{preprocessor}{#include <opencv2/highgui.hpp>}

\textcolor{keyword}{using namespace }\hyperlink{namespacestd}{std};
\textcolor{keyword}{using namespace }\hyperlink{namespacecv}{cv};
\textcolor{keyword}{using namespace }\hyperlink{namespacecanvascv}{canvascv};

\textcolor{keywordtype}{void} handlePolyPoints(\hyperlink{classcanvascv_1_1Canvas}{Canvas} &c, \hyperlink{classcanvascv_1_1Polygon}{Polygon} &poly)
\{
    vector<Point> polyPoints;
    poly.\hyperlink{classcanvascv_1_1Polygon_ada0df457225c06769d7a90d71f58ed7f}{getPoints}(polyPoints);
    stringstream s;
    s <<  \textcolor{stringliteral}{"polygon points are at :\(\backslash\)n["};
    \textcolor{keywordflow}{for} (\textcolor{keyword}{auto} &v : polyPoints)
    \{
        s << \textcolor{stringliteral}{" "} << v << \textcolor{stringliteral}{" "};
    \}
    s <<  \textcolor{stringliteral}{"]\(\backslash\)n\(\backslash\)n"};
    s <<  \textcolor{stringliteral}{"Use q to exit."};
    c.\hyperlink{classcanvascv_1_1Canvas_aaedea276b82a8a4cfc0895ae81113cfd}{setScreenText}(s.str());
\}

\textcolor{keywordtype}{int} main(\textcolor{keywordtype}{int} argc, \textcolor{keywordtype}{char} **argv)
\{
    --argc;
    ++argv;
    \textcolor{keywordflow}{if} (! argc)
    \{
        Canvas::fatal(\textcolor{stringliteral}{"Must get a path to an image as a parameter"} , -1);
    \}

    Mat image = imread(argv[0]);
    \textcolor{keywordflow}{if} (image.empty())
    \{
        Canvas::fatal(\textcolor{keywordtype}{string}(\textcolor{stringliteral}{"Cannot load image "}) + argv[0], -2);
    \}

    \hyperlink{classcanvascv_1_1Canvas}{Canvas} c(\textcolor{stringliteral}{"Shapes"}, image.size());
    c.\hyperlink{classcanvascv_1_1Canvas_ae68d3277e738d349232400b38f0e5f9e}{enableScreenText}();

    c.\hyperlink{classcanvascv_1_1Canvas_a402c43a42c0089c48a96e5303c1c1fe8}{enableStatusMsg}();
    c.\hyperlink{classcanvascv_1_1Canvas_a14828809edd29d789170284a86f16f23}{setDefaultStatusMsg}(\textcolor{stringliteral}{"Click-Release-Drag to create a single polygon.\(\backslash\)n"}
                          \textcolor{stringliteral}{"Select/Unselect with the mouse.\(\backslash\)n"}
                          \textcolor{stringliteral}{"Drag selected polygon with the mouse.\(\backslash\)n"}
                          \textcolor{stringliteral}{"DEL deleted the selected shape."});

    c.\hyperlink{classcanvascv_1_1Canvas_ac61735c6f4cb6a88d84331540ab25d39}{setShapeType}(Polygon::type); \textcolor{comment}{// default shape type for direct GUI creation}
    c.\hyperlink{classcanvascv_1_1Canvas_a64a459b16965e23de992cd2a301c68f4}{notifyOnShapeCreate}([&c](\hyperlink{classcanvascv_1_1Shape}{Shape}* shape)
    \{
        handlePolyPoints(c, *(\hyperlink{classcanvascv_1_1Polygon}{Polygon}*) shape); \textcolor{comment}{// it can only be a Polygon}
        c.\hyperlink{classcanvascv_1_1Canvas_ac61735c6f4cb6a88d84331540ab25d39}{setShapeType}(\textcolor{stringliteral}{""}); \textcolor{comment}{// default shape type for direct GUI creation}
    \});
    c.\hyperlink{classcanvascv_1_1Canvas_a5a6da8ae08b08a20d7fe15564bda5515}{notifyOnShapeModify}([&c](\hyperlink{classcanvascv_1_1Shape}{Shape}* shape)
    \{
        handlePolyPoints(c, *(\hyperlink{classcanvascv_1_1Polygon}{Polygon}*) shape);
    \});
    c.\hyperlink{classcanvascv_1_1Canvas_a1e7c26b39fd247e85941a6542f1b94c3}{notifyOnShapeDelete}([&c](\hyperlink{classcanvascv_1_1Shape}{Shape}*)
    \{
        c.\hyperlink{classcanvascv_1_1Canvas_ac61735c6f4cb6a88d84331540ab25d39}{setShapeType}(Polygon::type); \textcolor{comment}{// default shape type for direct GUI creation}
    \});

    namedWindow(\textcolor{stringliteral}{"Shapes"}, WINDOW\_AUTOSIZE);
    c.\hyperlink{classcanvascv_1_1Canvas_acf6e5d4b40aec610b0dc8c4f6bf93ac1}{setMouseCallback}(); \textcolor{comment}{// optional for mouse usage see also (example\_selectbox.cpp)}

    \textcolor{keywordtype}{int} key = 0;
    Mat out; \textcolor{comment}{// keeping it out of the loop is a little more efficient}
    \textcolor{keywordflow}{do}
    \{
        \textcolor{keywordflow}{switch} (key)
        \{
        \textcolor{keywordflow}{case} 65535:
            c.\hyperlink{classcanvascv_1_1Canvas_a2fb88addb88a21757d4272e64acd30ae}{deleteActive}();
            c.\hyperlink{classcanvascv_1_1Canvas_aaedea276b82a8a4cfc0895ae81113cfd}{setScreenText}(\textcolor{stringliteral}{""});
            \textcolor{keywordflow}{break};
        \}

        c.\hyperlink{classcanvascv_1_1Canvas_a018c66e277de7904b8146ea3f3feebdd}{redrawOn}(image, out);
        c.\hyperlink{classcanvascv_1_1Canvas_acaf9494a5668046dd0a8908aa97a7a43}{imshow}(out);
        key = c.\hyperlink{classcanvascv_1_1Canvas_a59397db05f5d9e45264f626f6a2ae528}{waitKeyEx}(); \textcolor{comment}{// GUI and callbacks happen here}
    \} \textcolor{keywordflow}{while} (key != \textcolor{charliteral}{'q'});

    destroyAllWindows();
    \textcolor{keywordflow}{return} 0;
\}
\end{DoxyCode}
 Notes\+:
\begin{DoxyItemize}
\item \hyperlink{classcanvascv_1_1Canvas_a402c43a42c0089c48a96e5303c1c1fe8}{canvascv\+::\+Canvas\+::enable\+Status\+Msg()} is optional, but recommended when using shapes and widgets. It takes a small portion of the screen area, but gives helpful information to the user.
\item \hyperlink{classcanvascv_1_1Canvas_a14828809edd29d789170284a86f16f23}{canvascv\+::\+Canvas\+::set\+Default\+Status\+Msg()} is optional. There are built in status messages that are displayed during shape creation and editing. When nothing is selected, then your default status message will be displayed.
\item The Canvas needs to know what shape to create when the user left clicks with the mouse. This is done with \hyperlink{classcanvascv_1_1Canvas_ac61735c6f4cb6a88d84331540ab25d39}{canvascv\+::\+Canvas\+::set\+Shape\+Type()}.
\item You can register for Canvas create/modify/delete notification on the Shapes that it creates/modify/deletes. This is the place to react to the user\textquotesingle{}s G\+UI actions.
\item When executed with a path to an image, this gives you (depends on your image)\+:  ~\newline

\end{DoxyItemize}\hypertarget{tutshapes_shapes_s2}{}\subsection{Creating a clickable rectangle from code}\label{tutshapes_shapes_s2}
As you know, in CV tracking is split into 2 phases -\/ detection and tracking.

Let\textquotesingle{}s assume we have a people detection code, and we\textquotesingle{}re detecting people in a croud.

We want to enable the user to choose a specific detection to start tracking. In the most simple form -\/ we want the rectangles we draw to do something when they are clicked.

In this tutorial we won\textquotesingle{}t track anything, but just output the chosen rectangle coordinates to the screen.

The base code we\textquotesingle{}ll be using is from {\itshape Open\+C\+V-\/3.\+2.\+0/samples/cpp/peopledetect.cpp}.

The modified code can be optimized for performance, but here we just want to show the {\itshape Canvas\+CV} way of doing it.

You are encouraged to diff the tutorial version and the Open\+CV version with your favorite diff tool, but the most important difference is this\+:


\begin{DoxyCode}
1 <         rectangle(img, r.tl(), r.br(), cv::Scalar(0,255,0), 3);
2 ---
3 >         shared\_ptr<Rectangle> shapeRect = canvas.createShape<Rectangle>();
4 >         shapeRect->setRect(cv::RotatedRect(Point(r.x + r.width / 2., r.y + r.height / 2.), r.size(), 0));
5 >         shapeRect->setOutlineColor(Colors::Green);
6 >         shapeRect->setThickness(3);
7 >         shapeRect->setLocked(true); // cannot be dragged
8 >         shapeRect->notifyOnEvent([&canvas](Shape *shape, Shape::Event evt)
9 >         \{
10 >             if (evt == Shape::SELECT)
11 >             \{
12 >                 canvas.setScreenText(CCV\_STR("User clicked in rect " <<
       ((Rectangle*)shape)->getRect().boundingRect()));
13 >             \}
14 >         \});
\end{DoxyCode}


In the code above, instead of drawing a rectangle, we create a \hyperlink{classcanvascv_1_1Rectangle}{canvascv\+::\+Rectangle} object on the {\itshape Canvas} and attach a callback to that specific instance.

The \hyperlink{classcanvascv_1_1Shape_a3fa381d7be3c6bb3cd736e237a444d5c}{canvascv\+::\+Shape\+::\+Event} we want is {\itshape S\+E\+L\+E\+CT}, which is when a shape is selected with the mouse (Clicked).

The full code is a little long because the original code is long\+:


\begin{DoxyCode}
\textcolor{preprocessor}{#include <iostream>}
\textcolor{preprocessor}{#include <stdexcept>}
\textcolor{preprocessor}{#include <opencv2/objdetect.hpp>}
\textcolor{preprocessor}{#include <opencv2/highgui.hpp>}
\textcolor{preprocessor}{#include <opencv2/imgproc.hpp>}
\textcolor{preprocessor}{#include <opencv2/imgcodecs.hpp>}
\textcolor{preprocessor}{#include <opencv2/video.hpp>}
\textcolor{preprocessor}{#include <opencv2/videoio.hpp>}

\textcolor{keyword}{using namespace }\hyperlink{namespacecv}{cv};
\textcolor{keyword}{using namespace }\hyperlink{namespacestd}{std};

\textcolor{comment}{// CanvasCV change#1 - include and namespace}
\textcolor{preprocessor}{#include <canvascv/canvas.h>}
\textcolor{preprocessor}{#include <canvascv/shapes/rectangle.h>}
\textcolor{keyword}{using namespace }\hyperlink{namespacecanvascv}{canvascv};


\textcolor{keyword}{const} \textcolor{keywordtype}{char}* keys =
\{
    \textcolor{stringliteral}{"\{ help h      |                     | print help message \}"}
    \textcolor{stringliteral}{"\{ image i     |                     | specify input image\}"}
    \textcolor{stringliteral}{"\{ camera c    |                     | enable camera capturing \}"}
    \textcolor{stringliteral}{"\{ video v     | ../data/vtest.avi   | use video as input \}"}
    \textcolor{stringliteral}{"\{ directory d |                     | images directory\}"}
\};

\textcolor{comment}{// CanvasCV change#8 - we draw on the canvas instead of the real img}
\textcolor{keyword}{static} \textcolor{keywordtype}{void} detectAndDraw(\textcolor{keyword}{const} HOGDescriptor &hog, Mat &img, \hyperlink{classcanvascv_1_1Canvas}{Canvas} &canvas)
\{
    vector<Rect> found, found\_filtered;
    \textcolor{keywordtype}{double} t = (double) getTickCount();
    \textcolor{comment}{// Run the detector with default parameters. to get a higher hit-rate}
    \textcolor{comment}{// (and more false alarms, respectively), decrease the hitThreshold and}
    \textcolor{comment}{// groupThreshold (set groupThreshold to 0 to turn off the grouping completely).}
    hog.detectMultiScale(img, found, 0, Size(8,8), Size(32,32), 1.05, 2);
    t = (double) getTickCount() - t;
    cout << \textcolor{stringliteral}{"detection time = "} << (t*1000./cv::getTickFrequency()) << \textcolor{stringliteral}{" ms"} << endl;

    \textcolor{keywordflow}{for}(\textcolor{keywordtype}{size\_t} i = 0; i < found.size(); i++ )
    \{
        Rect r = found[i];

        \textcolor{keywordtype}{size\_t} j;
        \textcolor{comment}{// Do not add small detections inside a bigger detection.}
        \textcolor{keywordflow}{for} ( j = 0; j < found.size(); j++ )
            \textcolor{keywordflow}{if} ( j != i && (r & found[j]) == r )
                \textcolor{keywordflow}{break};

        \textcolor{keywordflow}{if} ( j == found.size() )
            found\_filtered.push\_back(r);
    \}

    \textcolor{keywordflow}{for} (\textcolor{keywordtype}{size\_t} i = 0; i < found\_filtered.size(); i++)
    \{
        Rect r = found\_filtered[i];

        \textcolor{comment}{// The HOG detector returns slightly larger rectangles than the real objects,}
        \textcolor{comment}{// so we slightly shrink the rectangles to get a nicer output.}
        r.x += cvRound(r.width*0.1);
        r.width = cvRound(r.width*0.8);
        r.y += cvRound(r.height*0.07);
        r.height = cvRound(r.height*0.8);
        shared\_ptr<Rectangle> shapeRect = canvas.\hyperlink{classcanvascv_1_1Canvas_a630ac92458f1718d0c597e96dd5a4aef}{createShape}<
      \hyperlink{classcanvascv_1_1Rectangle}{Rectangle}>();
        shapeRect->\hyperlink{classcanvascv_1_1Rectangle_a5149d50c87c3388619ace3badd868f50}{setRect}(cv::RotatedRect(Point(r.x + r.width / 2., r.y + r.height / 2.), r.size(),
       0));
        shapeRect->setOutlineColor(Colors::Green);
        shapeRect->setThickness(3);
        shapeRect->setLocked(\textcolor{keyword}{true}); \textcolor{comment}{// cannot be dragged}
        shapeRect->notifyOnEvent([&canvas](\hyperlink{classcanvascv_1_1Shape}{Shape} *shape, \hyperlink{classcanvascv_1_1Shape_a3fa381d7be3c6bb3cd736e237a444d5c}{Shape::Event} evt)
        \{
            \textcolor{keywordflow}{if} (evt == Shape::SELECT)
            \{
                canvas.\hyperlink{classcanvascv_1_1Canvas_aaedea276b82a8a4cfc0895ae81113cfd}{setScreenText}(CCV\_STR(\textcolor{stringliteral}{"User clicked in rect "} << ((
      \hyperlink{classcanvascv_1_1Rectangle}{Rectangle}*)shape)->getRect().boundingRect()));
            \}
        \});
    \}
\}

\textcolor{keywordtype}{int} main(\textcolor{keywordtype}{int} argc, \textcolor{keywordtype}{char}** argv)
\{
    CommandLineParser parser(argc, argv, keys);

    \textcolor{keywordflow}{if} (parser.has(\textcolor{stringliteral}{"help"}))
    \{
        cout << \textcolor{stringliteral}{"\(\backslash\)nThis program demonstrates the use of the HoG descriptor using\(\backslash\)n"}
            \textcolor{stringliteral}{" HOGDescriptor::hog.setSVMDetector(HOGDescriptor::getDefaultPeopleDetector());\(\backslash\)n"};
        parser.printMessage();
        cout << \textcolor{stringliteral}{"During execution:\(\backslash\)n\(\backslash\)tHit q or ESC key to quit.\(\backslash\)n"}
            \textcolor{stringliteral}{"\(\backslash\)tUsing OpenCV version "} << CV\_VERSION << \textcolor{stringliteral}{"\(\backslash\)n"}
            \textcolor{stringliteral}{"Note: camera device number must be different from -1.\(\backslash\)n"} << endl;
        \textcolor{keywordflow}{return} 0;
    \}

    HOGDescriptor hog;
    hog.setSVMDetector(HOGDescriptor::getDefaultPeopleDetector());
    namedWindow(\textcolor{stringliteral}{"people detector"}, 1);

    \textcolor{comment}{// CanvasCV change#2 - attach a canvas to a window with mouse callbacks}
    \hyperlink{classcanvascv_1_1Canvas}{Canvas} canvas(\textcolor{stringliteral}{"people detector"});
    canvas.\hyperlink{classcanvascv_1_1Canvas_ae68d3277e738d349232400b38f0e5f9e}{enableScreenText}();
    canvas.\hyperlink{classcanvascv_1_1Canvas_acf6e5d4b40aec610b0dc8c4f6bf93ac1}{setMouseCallback}();

    \textcolor{keywordtype}{string} pattern\_glob = \textcolor{stringliteral}{""};
    \textcolor{keywordtype}{string} video\_filename = \textcolor{stringliteral}{"../data/vtest.avi"};
    \textcolor{keywordtype}{int} camera\_id = -1;
    \textcolor{keywordflow}{if} (parser.has(\textcolor{stringliteral}{"directory"}))
    \{
        pattern\_glob = parser.get<\textcolor{keywordtype}{string}>(\textcolor{stringliteral}{"directory"});
    \}
    \textcolor{keywordflow}{else} \textcolor{keywordflow}{if} (parser.has(\textcolor{stringliteral}{"image"}))
    \{
        pattern\_glob = parser.get<\textcolor{keywordtype}{string}>(\textcolor{stringliteral}{"image"});
    \}
    \textcolor{keywordflow}{else} \textcolor{keywordflow}{if} (parser.has(\textcolor{stringliteral}{"camera"}))
    \{
        camera\_id = parser.get<\textcolor{keywordtype}{int}>(\textcolor{stringliteral}{"camera"});
    \}
    \textcolor{keywordflow}{else} \textcolor{keywordflow}{if} (parser.has(\textcolor{stringliteral}{"video"}))
    \{
        video\_filename = parser.get<\textcolor{keywordtype}{string}>(\textcolor{stringliteral}{"video"});
    \}

    \textcolor{keywordflow}{if} (!pattern\_glob.empty() || camera\_id != -1 || !video\_filename.empty())
    \{
        \textcolor{comment}{//Read from input image files}
        vector<String> filenames;
        \textcolor{comment}{//Read from video file}
        VideoCapture vc;
        Mat frame;

        \textcolor{keywordflow}{if} (!pattern\_glob.empty())
        \{
            String folder(pattern\_glob);
            glob(folder, filenames);
        \}
        \textcolor{keywordflow}{else} \textcolor{keywordflow}{if} (camera\_id != -1)
        \{
            vc.open(camera\_id);
            \textcolor{keywordflow}{if} (!vc.isOpened())
            \{
                \textcolor{comment}{// CanvasCV change#3 - using fatal}
                Canvas::fatal(CCV\_STR(\textcolor{stringliteral}{"can't open camera: "} << camera\_id), -1);
            \}
        \}
        \textcolor{keywordflow}{else}
        \{
            vc.open(video\_filename.c\_str());
            \textcolor{comment}{// CanvasCV change#4 - using fatal}
            \textcolor{keywordflow}{if} (!vc.isOpened())
                Canvas::fatal(CCV\_STR(\textcolor{stringliteral}{"can't open video file: "} << video\_filename), -1);
        \}

        vector<String>::const\_iterator it\_image = filenames.begin();

        Mat out;
        \textcolor{keywordflow}{for} (;;)
        \{
            \textcolor{keywordflow}{if} (!pattern\_glob.empty())
            \{
                \textcolor{keywordtype}{bool} read\_image\_ok = \textcolor{keyword}{false};
                \textcolor{keywordflow}{for} (; it\_image != filenames.end(); ++it\_image)
                \{
                    cout << \textcolor{stringliteral}{"\(\backslash\)nRead: "} << *it\_image << endl;
                    \textcolor{comment}{// Read current image}
                    frame = imread(*it\_image);

                    \textcolor{keywordflow}{if} (!frame.empty())
                    \{
                        ++it\_image;
                        read\_image\_ok = \textcolor{keyword}{true};
                        \textcolor{keywordflow}{break};
                    \}
                \}

                \textcolor{comment}{//No more valid images}
                \textcolor{keywordflow}{if} (!read\_image\_ok)
                \{
                    \textcolor{comment}{//Release the image in order to exit the while loop}
                    frame.release();
                \}
            \}
            \textcolor{keywordflow}{else}
            \{
                vc >> frame;
            \}

            \textcolor{keywordflow}{if} (frame.empty())
                \textcolor{keywordflow}{break};

            \textcolor{comment}{// CanvasCV change#5 - clean & set canvas limits}
            canvas.\hyperlink{classcanvascv_1_1Canvas_a8aa6686036b8a1a006718ae62f44b6c2}{clearShapes}();
            canvas.\hyperlink{classcanvascv_1_1Canvas_ab9ffc28f7a21e1375da18cc4f03343ae}{setSize}(frame.size());

            detectAndDraw(hog, frame, canvas);

            \textcolor{comment}{// CanvasCV change#6 - draw shapes & widgets}
            canvas.\hyperlink{classcanvascv_1_1Canvas_a018c66e277de7904b8146ea3f3feebdd}{redrawOn}(frame, out);
            canvas.\hyperlink{classcanvascv_1_1Canvas_acaf9494a5668046dd0a8908aa97a7a43}{imshow}(out);

            \textcolor{comment}{// CanvasCV change#7 - get keyboard events from the canvas}
            \textcolor{keywordtype}{int} c = canvas.\hyperlink{classcanvascv_1_1Canvas_a59397db05f5d9e45264f626f6a2ae528}{waitKeyEx}( vc.isOpened() ? 30 : 0 ) & 255;
            \textcolor{keywordflow}{if} ( c == \textcolor{charliteral}{'q'} || c == \textcolor{charliteral}{'Q'} || c == 27)
                \textcolor{keywordflow}{break};
        \}
    \}

    \textcolor{keywordflow}{return} 0;
\}
\end{DoxyCode}
 Notes for the {\ttfamily //\+Canvas\+CV change\#...} comments above\+:
\begin{DoxyItemize}
\item Here the canvas is cleared at each frame. There are ways to optimize this, but first you\textquotesingle{}ll need to see if this affects your performance at all.
\item As a general rule, if the image size changes, then use \hyperlink{classcanvascv_1_1Canvas_ab9ffc28f7a21e1375da18cc4f03343ae}{canvascv\+::\+Canvas\+::set\+Size()} to adapt to the new size.
\item When using the Canvas\+CV, you create shapes and widgets on the \hyperlink{classcanvascv_1_1Canvas}{canvascv\+::\+Canvas} and not on the {\itshape Mat}.
\item \hyperlink{classcanvascv_1_1Canvas_a59397db05f5d9e45264f626f6a2ae528}{canvascv\+::\+Canvas\+::wait\+Key\+Ex()} knows to update it\textquotesingle{}s internal shapes and widgets even if you pass 0 as a blocking delay indicator.
\item {\itshape C\+C\+V\+\_\+\+S\+TR} lets you create a string as you would write into a stream.
\item When executed with --video=\char`\"{}\+Path to opencv-\/3.\+2.\+0/samples/data/vtest.\+avi\char`\"{}, you can get  ~\newline

\end{DoxyItemize}

{\bfseries That\textquotesingle{}s all for this tutorial} \hypertarget{tutpersistency}{}\section{Persisting shapes}\label{tutpersistency}
What if your user created the shapes you needed from him, and you want to save that setup?

No problem! shapes can be saved to a file and read back from that file.

Currently only the shapes of the entire {\itshape Canvas} can be written to a file and read from it.\hypertarget{tutpersistency_persistency_s1}{}\subsection{Write and read of the user Polygon}\label{tutpersistency_persistency_s1}
This is the same program as from the previous tutorial (\hyperlink{tutshapes_shapes_s1}{Creating a polygon from the G\+UI})

Adding only these lines to screen text\+:


\begin{DoxyCode}
s <<  \textcolor{stringliteral}{"Use s to save.\(\backslash\)n"};
s <<  \textcolor{stringliteral}{"Use l to load."};
\end{DoxyCode}


And these lines to the switch\+:


\begin{DoxyCode}
\textcolor{keywordflow}{case} \textcolor{charliteral}{'s'}:
    c.\hyperlink{classcanvascv_1_1Canvas_a494bb06b1a29232f05807f4a0e480ebb}{writeShapesToFile}(\textcolor{stringliteral}{"canvas.xml"});
    \textcolor{keywordflow}{break};
\textcolor{keywordflow}{case} \textcolor{charliteral}{'l'}:
    c.\hyperlink{classcanvascv_1_1Canvas_ab68000bb631c2fa7bb8863e746a8cff3}{readShapesFromFile}(\textcolor{stringliteral}{"canvas.xml"});
    \textcolor{keywordflow}{break};
\end{DoxyCode}


Which gives\+:


\begin{DoxyCode}
\textcolor{preprocessor}{#include <canvascv/canvas.h>}
\textcolor{preprocessor}{#include <canvascv/shapes/polygon.h>}

\textcolor{preprocessor}{#include <opencv2/highgui.hpp>}

\textcolor{keyword}{using namespace }\hyperlink{namespacestd}{std};
\textcolor{keyword}{using namespace }\hyperlink{namespacecv}{cv};
\textcolor{keyword}{using namespace }\hyperlink{namespacecanvascv}{canvascv};

\textcolor{keywordtype}{void} handlePolyPoints(\hyperlink{classcanvascv_1_1Canvas}{Canvas} &c, \hyperlink{classcanvascv_1_1Polygon}{Polygon} &poly)
\{
    vector<Point> polyPoints;
    poly.\hyperlink{classcanvascv_1_1Polygon_ada0df457225c06769d7a90d71f58ed7f}{getPoints}(polyPoints);
    stringstream s;
    s <<  \textcolor{stringliteral}{"polygon points are at :\(\backslash\)n["};
    \textcolor{keywordflow}{for} (\textcolor{keyword}{auto} &v : polyPoints)
    \{
        s << \textcolor{stringliteral}{" "} << v << \textcolor{stringliteral}{" "};
    \}
    s <<  \textcolor{stringliteral}{"]\(\backslash\)n\(\backslash\)n"};
    s <<  \textcolor{stringliteral}{"Use q to exit.\(\backslash\)n"};
    s <<  \textcolor{stringliteral}{"Use s to save.\(\backslash\)n"};
    s <<  \textcolor{stringliteral}{"Use l to load."};
    c.\hyperlink{classcanvascv_1_1Canvas_aaedea276b82a8a4cfc0895ae81113cfd}{setScreenText}(s.str());
\}

\textcolor{keywordtype}{int} main(\textcolor{keywordtype}{int} argc, \textcolor{keywordtype}{char} **argv)
\{
    --argc;
    ++argv;
    \textcolor{keywordflow}{if} (! argc)
    \{
        Canvas::fatal(\textcolor{stringliteral}{"Must get a path to an image as a parameter"} , -1);
    \}

    Mat image = imread(argv[0]);
    \textcolor{keywordflow}{if} (image.empty())
    \{
        Canvas::fatal(\textcolor{keywordtype}{string}(\textcolor{stringliteral}{"Cannot load image "}) + argv[0], -2);
    \}

    \hyperlink{classcanvascv_1_1Canvas}{Canvas} c(\textcolor{stringliteral}{"Persistent Shapes"}, image.size());
    c.\hyperlink{classcanvascv_1_1Canvas_ae68d3277e738d349232400b38f0e5f9e}{enableScreenText}();

    c.\hyperlink{classcanvascv_1_1Canvas_a402c43a42c0089c48a96e5303c1c1fe8}{enableStatusMsg}();
    c.\hyperlink{classcanvascv_1_1Canvas_a14828809edd29d789170284a86f16f23}{setDefaultStatusMsg}(\textcolor{stringliteral}{"Click-Release-Drag to create a single polygon.\(\backslash\)n"}
                          \textcolor{stringliteral}{"Select/Unselect with the mouse.\(\backslash\)n"}
                          \textcolor{stringliteral}{"Drag selected polygon with the mouse.\(\backslash\)n"}
                          \textcolor{stringliteral}{"DEL deleted the selected shape."});

    c.\hyperlink{classcanvascv_1_1Canvas_ac61735c6f4cb6a88d84331540ab25d39}{setShapeType}(Polygon::type); \textcolor{comment}{// default shape type for direct GUI creation}
    c.\hyperlink{classcanvascv_1_1Canvas_a64a459b16965e23de992cd2a301c68f4}{notifyOnShapeCreate}([&c](\hyperlink{classcanvascv_1_1Shape}{Shape}* shape)
    \{
        handlePolyPoints(c, *(\hyperlink{classcanvascv_1_1Polygon}{Polygon}*) shape); \textcolor{comment}{// it can only be a Polygon}
        c.\hyperlink{classcanvascv_1_1Canvas_ac61735c6f4cb6a88d84331540ab25d39}{setShapeType}(\textcolor{stringliteral}{""}); \textcolor{comment}{// default shape type for direct GUI creation}
    \});
    c.\hyperlink{classcanvascv_1_1Canvas_a5a6da8ae08b08a20d7fe15564bda5515}{notifyOnShapeModify}([&c](\hyperlink{classcanvascv_1_1Shape}{Shape}* shape)
    \{
        handlePolyPoints(c, *(\hyperlink{classcanvascv_1_1Polygon}{Polygon}*) shape);
    \});
    c.\hyperlink{classcanvascv_1_1Canvas_a1e7c26b39fd247e85941a6542f1b94c3}{notifyOnShapeDelete}([&c](\hyperlink{classcanvascv_1_1Shape}{Shape}*)
    \{
        c.\hyperlink{classcanvascv_1_1Canvas_ac61735c6f4cb6a88d84331540ab25d39}{setShapeType}(Polygon::type); \textcolor{comment}{// default shape type for direct GUI creation}
    \});

    namedWindow(\textcolor{stringliteral}{"Persistent Shapes"}, WINDOW\_AUTOSIZE);
    c.\hyperlink{classcanvascv_1_1Canvas_acf6e5d4b40aec610b0dc8c4f6bf93ac1}{setMouseCallback}(); \textcolor{comment}{// optional for mouse usage see also (example\_selectbox.cpp)}

    \textcolor{keywordtype}{int} key = 0;
    Mat out; \textcolor{comment}{// keeping it out of the loop is a little more efficient}
    \textcolor{keywordflow}{do}
    \{
        \textcolor{keywordflow}{switch} (key)
        \{
        \textcolor{keywordflow}{case} \textcolor{charliteral}{'s'}:
            c.\hyperlink{classcanvascv_1_1Canvas_a494bb06b1a29232f05807f4a0e480ebb}{writeShapesToFile}(\textcolor{stringliteral}{"canvas.xml"});
            \textcolor{keywordflow}{break};
        \textcolor{keywordflow}{case} \textcolor{charliteral}{'l'}:
            c.\hyperlink{classcanvascv_1_1Canvas_ab68000bb631c2fa7bb8863e746a8cff3}{readShapesFromFile}(\textcolor{stringliteral}{"canvas.xml"});
            \textcolor{keywordflow}{break};
        \textcolor{keywordflow}{case} 65535:
            c.\hyperlink{classcanvascv_1_1Canvas_a2fb88addb88a21757d4272e64acd30ae}{deleteActive}();
            c.\hyperlink{classcanvascv_1_1Canvas_aaedea276b82a8a4cfc0895ae81113cfd}{setScreenText}(\textcolor{stringliteral}{""});
            \textcolor{keywordflow}{break};
        \}

        c.\hyperlink{classcanvascv_1_1Canvas_a018c66e277de7904b8146ea3f3feebdd}{redrawOn}(image, out);
        c.\hyperlink{classcanvascv_1_1Canvas_acaf9494a5668046dd0a8908aa97a7a43}{imshow}(out);
        key = c.\hyperlink{classcanvascv_1_1Canvas_a59397db05f5d9e45264f626f6a2ae528}{waitKeyEx}(); \textcolor{comment}{// GUI and callbacks happen here}
    \} \textcolor{keywordflow}{while} (key != \textcolor{charliteral}{'q'});

    destroyAllWindows();
    \textcolor{keywordflow}{return} 0;
\}
\end{DoxyCode}
 Notes\+:
\begin{DoxyItemize}
\item The \hyperlink{classcanvascv_1_1Canvas_ab68000bb631c2fa7bb8863e746a8cff3}{canvascv\+::\+Canvas\+::read\+Shapes\+From\+File()} will clear itself from shapes before reading the shapes from the file.
\item When executed with a path to an image, this gives you (depends on your image)\+:  ~\newline

\end{DoxyItemize}

{\bfseries That\textquotesingle{}s all for this tutorial} \hypertarget{tutbuttons}{}\section{Using buttons}\label{tutbuttons}
A {\itshape Button} is one of the most a basic widgets.

It changes its contour when pressed (according to the current \hyperlink{classcanvascv_1_1Theme}{canvascv\+::\+Theme}).

It has text displayed on it and you attach a callback to know when it was pressed.\hypertarget{tutbuttons_buttons_s1}{}\subsection{Resizing the image with buttons}\label{tutbuttons_buttons_s1}
Up to now we\textquotesingle{}ve been using W\+I\+N\+D\+O\+W\+\_\+\+A\+U\+T\+O\+S\+I\+ZE to avoid stretching the image when creating the window.

Let\textquotesingle{}s write a small program which allows resizing of the image.

This will be done with 2 buttons \char`\"{}+\char`\"{} and \char`\"{}-\/\char`\"{} to increse and decrease the size of the image respectively.

Here we also make an effort to keep a button location aligned to a changing Canvas size. This won\textquotesingle{}t be needed when you\textquotesingle{}ll learn about the layout managers.

Read the code and see the notes after it\+:


\begin{DoxyCode}
\textcolor{preprocessor}{#include <canvascv/canvas.h>}
\textcolor{preprocessor}{#include <canvascv/widgets/button.h>}

\textcolor{keyword}{using namespace }\hyperlink{namespacecv}{cv};
\textcolor{keyword}{using namespace }\hyperlink{namespacecanvascv}{canvascv};

\textcolor{keywordtype}{int} main(\textcolor{keywordtype}{int} argc, \textcolor{keywordtype}{char} **argv)
\{
    --argc;
    ++argv;
    \textcolor{keywordflow}{if} (! argc)
    \{
        Canvas::fatal(\textcolor{stringliteral}{"Must get a path to an image as a parameter"} , -1);
    \}

    Mat orig = imread(argv[0]);
    \textcolor{keywordflow}{if} (orig.empty())
    \{
        Canvas::fatal(\textcolor{keywordtype}{string}(\textcolor{stringliteral}{"Cannot load image "}) + argv[0], -2);
    \}
    Mat image = orig;

    namedWindow(\textcolor{stringliteral}{"Buttons"}, WINDOW\_AUTOSIZE); \textcolor{comment}{// disable mouse resize}
    \hyperlink{classcanvascv_1_1Canvas}{Canvas} c(\textcolor{stringliteral}{"Buttons"}, orig.size());
    c.\hyperlink{classcanvascv_1_1Canvas_a402c43a42c0089c48a96e5303c1c1fe8}{enableStatusMsg}(); \textcolor{comment}{// shows hover over button msgs}
    c.\hyperlink{classcanvascv_1_1Canvas_acf6e5d4b40aec610b0dc8c4f6bf93ac1}{setMouseCallback}();

    \textcolor{keywordtype}{double} ratio = 1.;

    \textcolor{keyword}{auto} rightSideButton = Button::create(c,                                 \textcolor{comment}{// on the canvas}
                                          Point(image.cols - 20, 0),         \textcolor{comment}{// top-right aligned on the
       screen}
                                          \textcolor{stringliteral}{"+"},                               \textcolor{comment}{// displayed on button}
                                          \textcolor{stringliteral}{"increase size"},                   \textcolor{comment}{// displayed during mouse
       hover}
                                          [&c, &orig, &image, &ratio](Widget *w) \textcolor{comment}{// callback can be a C++11
       lambda}
    \{
        ratio += 0.05;
        cv::resize(orig, image, Size(), ratio, ratio);
        c.setImage(image);  \textcolor{comment}{// change the image during waitKeyEx(0)}
        w->setLocation(Point(image.cols - 20, 0)); \textcolor{comment}{// keep aligned to top right without a layout manager}
    \}
    );

    Button::create(c,                               \textcolor{comment}{// on the canvas}
                   Point(0, 0),                     \textcolor{comment}{// top-left aligned on the screen}
                   \textcolor{stringliteral}{"-"},                             \textcolor{comment}{// displayed on button}
                   \textcolor{stringliteral}{"decrease size"},                 \textcolor{comment}{// displayed during mouse hover}
                   [&c, &orig, &image, &ratio, rightSideButton](Widget*) \textcolor{comment}{// callback can be a C++11 lambda}
    \{
        ratio -= 0.05;
        cv::resize(orig, image, Size(), ratio, ratio);
        c.\hyperlink{classcanvascv_1_1Canvas_a441c5882c7ebebd454a306b3c3478ae7}{setImage}(image);  \textcolor{comment}{// change the image during waitKeyEx(0)}
        rightSideButton->setLocation(Point(image.cols - 20, 0)); \textcolor{comment}{// keep aligned to top right without a
       layout manager}
    \}
    );

    \textcolor{keywordtype}{int} key = 0;
    Mat out;
    \textcolor{keywordflow}{do}
    \{
        c.\hyperlink{classcanvascv_1_1Canvas_a018c66e277de7904b8146ea3f3feebdd}{redrawOn}(image, out);  \textcolor{comment}{// draw the canvas on the image copy}
        c.\hyperlink{classcanvascv_1_1Canvas_acaf9494a5668046dd0a8908aa97a7a43}{imshow}(out);
        key = c.\hyperlink{classcanvascv_1_1Canvas_a59397db05f5d9e45264f626f6a2ae528}{waitKeyEx}(); \textcolor{comment}{// GUI and callbacks happen here}

    \} \textcolor{keywordflow}{while} (key != \textcolor{charliteral}{'q'});

    destroyAllWindows();

    \textcolor{keywordflow}{return} 0;
\}
\end{DoxyCode}
 Notes\+:
\begin{DoxyItemize}
\item We only need a variable to widgets we want to acces by code, so only the {\itshape right\+Side\+Button} has a variable, since we\textquotesingle{}re manually aligning it to the right on resizing.
\item For a single image display, the best practice is to use \hyperlink{classcanvascv_1_1Canvas_a59397db05f5d9e45264f626f6a2ae528}{canvascv\+::\+Canvas\+::wait\+Key\+Ex()} (default delay is 0).
\begin{DoxyItemize}
\item Since during a zero delay, the Canvas is self updating, it needs to know that the image has changed, and this is what \hyperlink{classcanvascv_1_1Canvas_a441c5882c7ebebd454a306b3c3478ae7}{canvascv\+::\+Canvas\+::set\+Image()} is for.
\end{DoxyItemize}
\item We\textquotesingle{}re using {\itshape enable\+Status\+Msg} since we want the mouse hover over the buttons to show our status message for it.
\item This tutorial is using C++11 lambda expressions as callbacks, but anything which has the \char`\"{}void(\+Widget$\ast$)\char`\"{} signature will work here.
\item Here is an image decreased by this code\+:  ~\newline

\end{DoxyItemize}\hypertarget{tutbuttons_buttons_s2}{}\subsection{Text alignment with buttons}\label{tutbuttons_buttons_s2}
This will be more relevant on the {\itshape layout managers} tutorial, later on.

Currently, lets assume we have a button with 2 lines of text, a short line and below it a long line.

We could want the text to be centered, or aligned to left, or aligned to right.

The way to do this is with anchoring to the flow. If you set your anchor at some point, then the text can\textquotesingle{}t move the anchor and flows to the other side. By doing that, it is aligned to the anchored side.

See this code\+: 
\begin{DoxyCode}
\textcolor{preprocessor}{#include <canvascv/canvas.h>}
\textcolor{preprocessor}{#include <canvascv/widgets/button.h>}

\textcolor{keyword}{using namespace }\hyperlink{namespacecv}{cv};
\textcolor{keyword}{using namespace }\hyperlink{namespacecanvascv}{canvascv};

\textcolor{keywordtype}{int} main(\textcolor{keywordtype}{int} argc, \textcolor{keywordtype}{char} **argv)
\{
    --argc;
    ++argv;
    \textcolor{keywordflow}{if} (! argc)
    \{
        Canvas::fatal(\textcolor{stringliteral}{"Must get a path to an image as a parameter"} , -1);
    \}

    Mat image = imread(argv[0]);
    \textcolor{keywordflow}{if} (image.empty())
    \{
        Canvas::fatal(\textcolor{keywordtype}{string}(\textcolor{stringliteral}{"Cannot load image "}) + argv[0], -2);
    \}

    namedWindow(\textcolor{stringliteral}{"ButTextAlign"}, WINDOW\_AUTOSIZE); \textcolor{comment}{// disable mouse resize}
    \hyperlink{classcanvascv_1_1Canvas}{Canvas} c(\textcolor{stringliteral}{"ButTextAlign"}, image.size());
    c.\hyperlink{classcanvascv_1_1Canvas_a402c43a42c0089c48a96e5303c1c1fe8}{enableStatusMsg}(); \textcolor{comment}{// shows hover over button msgs}
    c.\hyperlink{classcanvascv_1_1Canvas_acf6e5d4b40aec610b0dc8c4f6bf93ac1}{setMouseCallback}();

    \textcolor{keywordtype}{string} buttonText = \textcolor{stringliteral}{"short\(\backslash\)n"}
                        \textcolor{stringliteral}{"a longer line"};
    \textcolor{keyword}{auto} b1 = Button::create(c,               \textcolor{comment}{// on the canvas}
                             Point(10,10),    \textcolor{comment}{// top-right aligned on the screen}
                             buttonText,      \textcolor{comment}{// displayed on button}
                             \textcolor{stringliteral}{"centered"});     \textcolor{comment}{// displayed during mouse hover}
    b1->setFlowAnchor(Widget::CENTER);

    \textcolor{keyword}{auto} b2 = Button::create(c,               \textcolor{comment}{// on the canvas}
                             Point(10,110),    \textcolor{comment}{// top-right aligned on the screen}
                             buttonText,      \textcolor{comment}{// displayed on button}
                             \textcolor{stringliteral}{"left aligned"}); \textcolor{comment}{// displayed during mouse hover}
    b2->setFlowAnchor(Widget::LEFT);

    \textcolor{keyword}{auto} b3 = Button::create(c,               \textcolor{comment}{// on the canvas}
                             Point(10,210),    \textcolor{comment}{// top-right aligned on the screen}
                             buttonText,      \textcolor{comment}{// displayed on button}
                             \textcolor{stringliteral}{"right aligned"});\textcolor{comment}{// displayed during mouse hover}
    b3->setFlowAnchor(Widget::RIGHT);

    \textcolor{keywordtype}{int} key = 0;
    Mat out;
    \textcolor{keywordflow}{do}
    \{
        c.\hyperlink{classcanvascv_1_1Canvas_a018c66e277de7904b8146ea3f3feebdd}{redrawOn}(image, out);  \textcolor{comment}{// draw the canvas on the image copy}
        c.\hyperlink{classcanvascv_1_1Canvas_acaf9494a5668046dd0a8908aa97a7a43}{imshow}(out);
        key = c.\hyperlink{classcanvascv_1_1Canvas_a59397db05f5d9e45264f626f6a2ae528}{waitKeyEx}(); \textcolor{comment}{// GUI and callbacks happen here}

    \} \textcolor{keywordflow}{while} (key != \textcolor{charliteral}{'q'});

    destroyAllWindows();

    \textcolor{keywordflow}{return} 0;
\}
\end{DoxyCode}
 Notes\+:
\begin{DoxyItemize}
\item the \hyperlink{classcanvascv_1_1Widget_a69f455f7fbf67f636ee1155795057b87}{canvascv\+::\+Widget\+::set\+Flow\+Anchor()} does the work here.
\item Here is a possible outcome this code\+:  ~\newline

\end{DoxyItemize}

{\bfseries That\textquotesingle{}s all for this tutorial} \hypertarget{tutcheckbox}{}\section{Using Check\+Boxes (and Radio\+Buttons and a Selection\+Box)}\label{tutcheckbox}
Giving the user multiple options and collecting them is common to \hyperlink{classcanvascv_1_1CheckBoxes}{canvascv\+::\+Check\+Boxes}, \hyperlink{classcanvascv_1_1RadioButtons}{canvascv\+::\+Radio\+Buttons} and \hyperlink{classcanvascv_1_1SelectionBox}{canvascv\+::\+Selection\+Box}.

We going to take a look at using the {\itshape Check\+Boxes} now. The concept is similar to {\itshape Radio\+Buttons} and the {\itshape Selection\+Box}. In the tutorial about layout managers we\textquotesingle{}ll see an example frame with several of them together.\hypertarget{tutcheckbox_checkbox_s1}{}\subsection{Check\+Boxes with a callback}\label{tutcheckbox_checkbox_s1}
This is the normal and preffered way of working with widgets.

The {\itshape Check\+Box} alows you to register a \hyperlink{classcanvascv_1_1Widget_a977cbd39cf203c5866f07f3645c7e4bc}{canvascv\+::\+Widget\+::\+C\+B\+User\+Selection} callback, which will be called on user altering the widget selections.

You get the widget and an index to the item that changed. It is then up to you to inspect the widget at that index (or other indexes).

See this code example, and read the notes after it\+: 
\begin{DoxyCode}
\textcolor{preprocessor}{#include <canvascv/canvas.h>}
\textcolor{preprocessor}{#include <canvascv/widgets/checkboxes.h>}

\textcolor{keyword}{using namespace }\hyperlink{namespacecanvascv}{canvascv};

\textcolor{keywordtype}{int} main(\textcolor{keywordtype}{int} argc, \textcolor{keywordtype}{char} **argv)
\{
    --argc;
    ++argv;
    \textcolor{keywordflow}{if} (! argc)
    \{
        \hyperlink{classcanvascv_1_1Canvas_add93c0d5cc1e9b49f97510952a8a1961}{Canvas::fatal}(\textcolor{stringliteral}{"Must get a path to an image as a parameter"} , -1);
    \}

    Mat image = imread(argv[0]);
    \textcolor{keywordflow}{if} (image.empty())
    \{
        \hyperlink{classcanvascv_1_1Canvas_add93c0d5cc1e9b49f97510952a8a1961}{Canvas::fatal}(\textcolor{keywordtype}{string}(\textcolor{stringliteral}{"Cannot load image "}) + argv[0], -2);
    \}

    \hyperlink{classcanvascv_1_1Canvas}{Canvas} c(\textcolor{stringliteral}{"CheckBoxes"}, image.size());
    c.\hyperlink{classcanvascv_1_1Canvas_ae68d3277e738d349232400b38f0e5f9e}{enableScreenText}();
    c.\hyperlink{classcanvascv_1_1Canvas_aaedea276b82a8a4cfc0895ae81113cfd}{setScreenText}(\textcolor{stringliteral}{"Use q to exit"});

    namedWindow(\textcolor{stringliteral}{"CheckBoxes"}, WINDOW\_AUTOSIZE);

    c.\hyperlink{classcanvascv_1_1Canvas_acf6e5d4b40aec610b0dc8c4f6bf93ac1}{setMouseCallback}(); \textcolor{comment}{// optional for mouse usage (see also example\_selectbox.cpp)}

    \hyperlink{classcanvascv_1_1CheckBoxes_a5108f52385a5cb19ad7fe52a18a91df0}{CheckBoxes::create}(c, \{
                           \textcolor{stringliteral}{"Option1"},
                           \textcolor{stringliteral}{"Option2"},
                           \textcolor{stringliteral}{"Option3"}
                       \},
                       [&c](Widget *w, int)
    \{
        \hyperlink{classcanvascv_1_1CheckBoxes}{CheckBoxes} *checkBoxes = (\hyperlink{classcanvascv_1_1CheckBoxes}{CheckBoxes} *) w;
        c.\hyperlink{classcanvascv_1_1Canvas_aaedea276b82a8a4cfc0895ae81113cfd}{setScreenText}(CCV\_STR(
                            \textcolor{stringliteral}{"Option1:"} <<  checkBoxes->\hyperlink{classcanvascv_1_1CheckBoxes_a2e544c7f81248c6b297460be5852506e}{isChecked}(0) << \textcolor{stringliteral}{"\(\backslash\)n"} <<
                            \textcolor{stringliteral}{"Option2:"} <<  checkBoxes->\hyperlink{classcanvascv_1_1CheckBoxes_a2e544c7f81248c6b297460be5852506e}{isChecked}(1) << \textcolor{stringliteral}{"\(\backslash\)n"} <<
                            \textcolor{stringliteral}{"Option3:"} <<  checkBoxes->\hyperlink{classcanvascv_1_1CheckBoxes_a2e544c7f81248c6b297460be5852506e}{isChecked}(2) << \textcolor{stringliteral}{"\(\backslash\)n\(\backslash\)n"} <<
                            \textcolor{stringliteral}{"Use q to exit"}
                            ));
    \},
    \{image.cols / 2, image.rows / 2\});

    \textcolor{keywordtype}{int} key = 0;
    c.\hyperlink{classcanvascv_1_1Canvas_a441c5882c7ebebd454a306b3c3478ae7}{setImage}(image);
    \textcolor{keywordflow}{do}
    \{
        key = c.\hyperlink{classcanvascv_1_1Canvas_a59397db05f5d9e45264f626f6a2ae528}{waitKeyEx}(); \textcolor{comment}{// GUI and callbacks happen here}
    \} \textcolor{keywordflow}{while} (key != \textcolor{charliteral}{'q'});

    destroyAllWindows();

    \textcolor{keywordflow}{return} 0;
\}
\end{DoxyCode}
 Notes\+:
\begin{DoxyItemize}
\item {\itshape C\+C\+V\+\_\+\+S\+TR} lets you create a string as you would write into a stream.
\item This tutorial is using C++11 lambda expressions as callbacks, but anything which has the \char`\"{}void(\+Widget$\ast$,int)\char`\"{} signature will work.
\item Note the alternate way of displaying images
\begin{DoxyItemize}
\item use \hyperlink{classcanvascv_1_1Canvas_a441c5882c7ebebd454a306b3c3478ae7}{canvascv\+::\+Canvas\+::set\+Image()} for the image you\textquotesingle{}re showing.
\item use \hyperlink{classcanvascv_1_1Canvas_a59397db05f5d9e45264f626f6a2ae528}{canvascv\+::\+Canvas\+::wait\+Key\+Ex()} without arguments to refresh widgets and handle internal events with your image as the background.
\end{DoxyItemize}
\item Here is a possible outcome of this code\+:  ~\newline

\end{DoxyItemize}\hypertarget{tutcheckbox_checkbox_s2}{}\subsection{Check\+Boxes with polling}\label{tutcheckbox_checkbox_s2}
Using polling might serve you right for some reason...

Instead of using a callback, you can check the values on each iteration.

This is not the recommended way.

See this code and read the notes after it\+: 
\begin{DoxyCode}
\textcolor{preprocessor}{#include <canvascv/canvas.h>}
\textcolor{preprocessor}{#include <canvascv/widgets/checkboxes.h>}

\textcolor{keyword}{using namespace }\hyperlink{namespacecanvascv}{canvascv};

\textcolor{keywordtype}{int} main(\textcolor{keywordtype}{int} argc, \textcolor{keywordtype}{char} **argv)
\{
    --argc;
    ++argv;
    \textcolor{keywordflow}{if} (! argc)
    \{
        \hyperlink{classcanvascv_1_1Canvas_add93c0d5cc1e9b49f97510952a8a1961}{Canvas::fatal}(\textcolor{stringliteral}{"Must get a path to an image as a parameter"} , -1);
    \}

    Mat image = imread(argv[0]);
    \textcolor{keywordflow}{if} (image.empty())
    \{
        \hyperlink{classcanvascv_1_1Canvas_add93c0d5cc1e9b49f97510952a8a1961}{Canvas::fatal}(\textcolor{keywordtype}{string}(\textcolor{stringliteral}{"Cannot load image "}) + argv[0], -2);
    \}

    \hyperlink{classcanvascv_1_1Canvas}{Canvas} c(\textcolor{stringliteral}{"CheckBoxes"}, image.size());

    namedWindow(\textcolor{stringliteral}{"CheckBoxes"}, WINDOW\_AUTOSIZE);

    c.\hyperlink{classcanvascv_1_1Canvas_acf6e5d4b40aec610b0dc8c4f6bf93ac1}{setMouseCallback}(); \textcolor{comment}{// optional for mouse usage (see also example\_selectbox.cpp)}

    \textcolor{keyword}{auto} checkBox = \hyperlink{classcanvascv_1_1CheckBoxes_a5108f52385a5cb19ad7fe52a18a91df0}{CheckBoxes::create}(c, \{
                                           \textcolor{stringliteral}{"Option1"}, \textcolor{comment}{// index 0}
                                           \textcolor{stringliteral}{"Option2"}, \textcolor{comment}{// index 1}
                                           \textcolor{stringliteral}{"Exit"}     \textcolor{comment}{// index 2}
                                       \},
                                       \hyperlink{classcanvascv_1_1Widget_a977cbd39cf203c5866f07f3645c7e4bc}{Widget::CBUserSelection}(), \textcolor{comment}{// empty callback}
                                       Point(image.cols / 2, image.rows / 2));

    \textcolor{keywordtype}{int} key = 0;
    \textcolor{keywordtype}{int} delay = 1000 / 25; \textcolor{comment}{// must use a delay when polling}
    Mat out;
    \textcolor{keywordflow}{do}
    \{
        c.\hyperlink{classcanvascv_1_1Canvas_a018c66e277de7904b8146ea3f3feebdd}{redrawOn}(image, out);
        c.\hyperlink{classcanvascv_1_1Canvas_acaf9494a5668046dd0a8908aa97a7a43}{imshow}(out);
        key = c.\hyperlink{classcanvascv_1_1Canvas_a59397db05f5d9e45264f626f6a2ae528}{waitKeyEx}(delay); \textcolor{comment}{// GUI and callbacks happen here}
    \} \textcolor{keywordflow}{while} (! checkBox->isChecked(2));

    destroyAllWindows();

    \textcolor{keywordflow}{return} 0;
\}
\end{DoxyCode}
 Notes\+:
\begin{DoxyItemize}
\item When polling is used we need all these stages in the main loop -\/
\begin{DoxyItemize}
\item \hyperlink{classcanvascv_1_1Canvas_a018c66e277de7904b8146ea3f3feebdd}{canvascv\+::\+Canvas\+::redraw\+On()}
\item \hyperlink{classcanvascv_1_1Canvas_acaf9494a5668046dd0a8908aa97a7a43}{canvascv\+::\+Canvas\+::imshow()}
\item \hyperlink{classcanvascv_1_1Canvas_a59397db05f5d9e45264f626f6a2ae528}{canvascv\+::\+Canvas\+::wait\+Key\+Ex()} with a non zero delay
\end{DoxyItemize}
\item Here is a possible outcome of this code\+:  ~\newline

\end{DoxyItemize}

{\bfseries That\textquotesingle{}s all for this tutorial} \hypertarget{tutlayouts}{}\section{Auto widget layouts}\label{tutlayouts}
In the previous tutorials we\textquotesingle{}ve positioned the Widgets independetly.

In many cases you want a set of widgets to be located together in a certain order.

For this you need Auto\+Layout derived classes\+:
\begin{DoxyItemize}
\item \hyperlink{classcanvascv_1_1HorizontalLayout}{canvascv\+::\+Horizontal\+Layout}
\item \hyperlink{classcanvascv_1_1VerticalLayout}{canvascv\+::\+Vertical\+Layout}
\item \hyperlink{classcanvascv_1_1HFrame}{canvascv\+::\+H\+Frame}
\item \hyperlink{classcanvascv_1_1VFrame}{canvascv\+::\+V\+Frame}
\end{DoxyItemize}

In this tutorial we\textquotesingle{}ll return to some previous tutorial code and see how to do it in a layout. Then we\textquotesingle{}ll write a simple form which gets inputs from the user.

Notice along the tutorial the usage of anchors and and stretching.

For anchors, you have 2 different A\+P\+Is\+:
\begin{DoxyItemize}
\item \hyperlink{classcanvascv_1_1Widget_a69f455f7fbf67f636ee1155795057b87}{canvascv\+::\+Widget\+::set\+Flow\+Anchor()} -\/ used internally by the widget to determine into which direction to grow.
\item \hyperlink{classcanvascv_1_1Widget_a7180ca0054874c853e805e7842741596}{canvascv\+::\+Widget\+::set\+Layout\+Anchor()}-\/ used by the layout holding this widget for alignments.
\end{DoxyItemize}

For stretching, you have 2 different A\+P\+Is\+:
\begin{DoxyItemize}
\item \hyperlink{classcanvascv_1_1Widget_a7cdddebd755c499712793f727a057733}{canvascv\+::\+Widget\+::set\+Stretch\+X()} and \hyperlink{classcanvascv_1_1Widget_a3ef50b76d33c332cea4e632346b6df33}{canvascv\+::\+Widget\+::set\+Stretch\+Y()}, which tells if we want to be equal in width/height to the largest widget in our layout.
\item \hyperlink{classcanvascv_1_1Widget_a7b1ed6190950de22565c244f5aac49a4}{canvascv\+::\+Widget\+::set\+Stretch\+X\+To\+Parent()} and \hyperlink{classcanvascv_1_1Widget_a825028e2405bfdf9d92d372e60585703}{canvascv\+::\+Widget\+::set\+Stretch\+Y\+To\+Parent()}, which tells if we want to be streched to the full width/height of our layout.
\end{DoxyItemize}\hypertarget{tutlayouts_layouts_s1}{}\subsection{Center text on the screen}\label{tutlayouts_layouts_s1}
In the tutorial \hyperlink{tutscreentext_screentext_s2_3}{Displaying text where ever you want} we created a {\itshape Text} widget and manually tried to center it on the screen.

To center it vertically and horizontally, you would use 2 layout managers -\/ \hyperlink{classcanvascv_1_1VerticalLayout}{canvascv\+::\+Vertical\+Layout} and \hyperlink{classcanvascv_1_1HorizontalLayout}{canvascv\+::\+Horizontal\+Layout}.

The top layout should stretch to the full dimensions of the screen (because we want {\itshape C\+E\+N\+T\+ER} to be relative to that). Here we choose a {\itshape Vertical\+Layout}, in which widgets are layered automatically veritcally, but horizontally you can specify where to put your widget.

Now we\textquotesingle{}ll add a {\itshape Horizontal\+Layout} to the above layout and request to be anchored to the {\itshape C\+E\+N\+T\+ER}.

In the {\itshape Horizontal\+Layout}, widgets are layered automatically horizontally, but vertically you can specify where to put your widget. This widget will strectch in the Y direction, so {\itshape C\+E\+N\+T\+ER} in it will be relative to the height of the image.

So finally we\textquotesingle{}ll add our {\itshape Text}, just like before, but now instead of adding it to the {\itshape Canvas}, we\textquotesingle{}ll add it to the above {\itshape Horizontal\+Layout}, and ask to be anchored to the {\itshape C\+E\+N\+T\+ER} of it.

So the {\itshape Vertical\+Layout} will {\itshape C\+E\+N\+T\+ER} the {\itshape Horizontal\+Layout} and it will {\itshape C\+E\+N\+T\+ER} the {\itshape Text}. Since the layouts are stretching to the full dimension of the image, it will always be centered.

The added/changed code is much shorter than the explanation\+: 
\begin{DoxyCode}
\textcolor{keyword}{using namespace }\hyperlink{namespacecanvascv}{canvascv};

\textcolor{comment}{// ...}

\textcolor{keyword}{auto} vLayout = \hyperlink{classcanvascv_1_1VerticalLayout_a3e0ae249db062663815d2d822e178dd3}{VerticalLayout::create}(c);
vLayout->setStretchXToParent(\textcolor{keyword}{true});
vLayout->setStretchYToParent(\textcolor{keyword}{true});

\textcolor{keyword}{auto} hLayout = \hyperlink{classcanvascv_1_1HorizontalLayout_aea31dd787cbf985687ead6a55efa1839}{HorizontalLayout::create}(*vLayout);
hLayout->setLayoutAnchor(\hyperlink{classcanvascv_1_1Widget_a98ca3c300ba50b316fa5a1d300456abea2cd62693af40f3c5f559a07d6a61a55d}{Widget::CENTER});
hLayout->setStretchYToParent(\textcolor{keyword}{true});

\textcolor{keyword}{auto} txt = \hyperlink{classcanvascv_1_1Text_a7f3552263b6f185f78d90549e7ac38f7}{Text::create}(*hLayout, \textcolor{stringliteral}{"Target Acquired!"});
txt->setLayoutAnchor(\hyperlink{classcanvascv_1_1Widget_a98ca3c300ba50b316fa5a1d300456abea2cd62693af40f3c5f559a07d6a61a55d}{Widget::CENTER});

\textcolor{comment}{// ...}
\end{DoxyCode}


This is the full code\+: 
\begin{DoxyCode}
\textcolor{preprocessor}{#include <canvascv/canvas.h>}
\textcolor{preprocessor}{#include <canvascv/widgets/text.h>}
\textcolor{preprocessor}{#include <canvascv/widgets/verticallayout.h>}
\textcolor{preprocessor}{#include <canvascv/widgets/horizontallayout.h>}

\textcolor{keyword}{using namespace }\hyperlink{namespacecanvascv}{canvascv};

\textcolor{keywordtype}{void} help(\hyperlink{classcanvascv_1_1Canvas}{Canvas} &c)
\{
    \textcolor{keyword}{static} \textcolor{keywordtype}{bool} showHelp = \textcolor{keyword}{true};
    \textcolor{keyword}{static} \textcolor{keywordtype}{string} helpMsg =
            \textcolor{stringliteral}{"Usage:\(\backslash\)n"}
            \textcolor{stringliteral}{"=====\(\backslash\)n"}
            \textcolor{stringliteral}{"h: toggle usage message\(\backslash\)n"}
            \textcolor{stringliteral}{"*: toggle canvas on/off\(\backslash\)n"}
            \textcolor{stringliteral}{"q: exit"};


    \textcolor{keywordflow}{if} (showHelp) c.\hyperlink{classcanvascv_1_1Canvas_aaedea276b82a8a4cfc0895ae81113cfd}{setScreenText}(helpMsg);
    \textcolor{keywordflow}{else} c.\hyperlink{classcanvascv_1_1Canvas_aaedea276b82a8a4cfc0895ae81113cfd}{setScreenText}(\textcolor{stringliteral}{""});

    showHelp = ! showHelp;
\}

\textcolor{keywordtype}{int} main(\textcolor{keywordtype}{int} argc, \textcolor{keywordtype}{char} **argv)
\{
    --argc;
    ++argv;
    \textcolor{keywordflow}{if} (! argc)
    \{
        \hyperlink{classcanvascv_1_1Canvas_add93c0d5cc1e9b49f97510952a8a1961}{Canvas::fatal}(\textcolor{stringliteral}{"Must get a path to an image as a parameter"} , -1);
    \}

    Mat image = imread(argv[0]);
    \textcolor{keywordflow}{if} (image.empty())
    \{
        \hyperlink{classcanvascv_1_1Canvas_add93c0d5cc1e9b49f97510952a8a1961}{Canvas::fatal}(\textcolor{keywordtype}{string}(\textcolor{stringliteral}{"Cannot load image "}) + argv[0], -2);
    \}

    \hyperlink{classcanvascv_1_1Canvas}{Canvas} c(\textcolor{stringliteral}{"LayoutTxt"}, image.size());
    c.\hyperlink{classcanvascv_1_1Canvas_ae68d3277e738d349232400b38f0e5f9e}{enableScreenText}(); \textcolor{comment}{// see it's documentation}

    help(c);

    namedWindow(\textcolor{stringliteral}{"LayoutTxt"}, WINDOW\_AUTOSIZE); \textcolor{comment}{// disable mouse resize}

    \textcolor{keyword}{auto} vLayout = \hyperlink{classcanvascv_1_1VerticalLayout_a3e0ae249db062663815d2d822e178dd3}{VerticalLayout::create}(c);
    vLayout->setStretchXToParent(\textcolor{keyword}{true});
    vLayout->setStretchYToParent(\textcolor{keyword}{true});

    \textcolor{keyword}{auto} hLayout = \hyperlink{classcanvascv_1_1HorizontalLayout_aea31dd787cbf985687ead6a55efa1839}{HorizontalLayout::create}(*vLayout);
    hLayout->setLayoutAnchor(\hyperlink{classcanvascv_1_1Widget_a98ca3c300ba50b316fa5a1d300456abea2cd62693af40f3c5f559a07d6a61a55d}{Widget::CENTER});
    hLayout->setStretchYToParent(\textcolor{keyword}{true});

    \textcolor{keyword}{auto} txt = \hyperlink{classcanvascv_1_1Text_a7f3552263b6f185f78d90549e7ac38f7}{Text::create}(*hLayout, \textcolor{stringliteral}{"Target Acquired!"});
    txt->setLayoutAnchor(\hyperlink{classcanvascv_1_1Widget_a98ca3c300ba50b316fa5a1d300456abea2cd62693af40f3c5f559a07d6a61a55d}{Widget::CENTER});
    txt->setFontHeight(50);
    txt->setOutlineColor(\hyperlink{classcanvascv_1_1Colors_a10aff24c53edf45b038d0636b061f9c2}{Colors::Red});
    txt->setThickness(2);

    \textcolor{keywordtype}{int} key = 0;
    Mat out;
    \textcolor{keywordflow}{do}
    \{
        \textcolor{keywordflow}{switch} (key)
        \{
        \textcolor{keywordflow}{case} \textcolor{charliteral}{'h'}:
            help(c);
            \textcolor{keywordflow}{break};
        \textcolor{keywordflow}{case} \textcolor{charliteral}{'*'}:
            c.\hyperlink{classcanvascv_1_1Canvas_aba149ea25c6cdad2673133a060355954}{setOn}(! c.\hyperlink{classcanvascv_1_1Canvas_afe6a2955a5bbee8903350b4fba3f4473}{getOn}());
            \textcolor{keywordflow}{break};
        \}

        c.\hyperlink{classcanvascv_1_1Canvas_a018c66e277de7904b8146ea3f3feebdd}{redrawOn}(image, out);  \textcolor{comment}{// draw the canvas on the image copy}

        imshow(\textcolor{stringliteral}{"LayoutTxt"}, out);

        key = c.\hyperlink{classcanvascv_1_1Canvas_a59397db05f5d9e45264f626f6a2ae528}{waitKeyEx}(); \textcolor{comment}{// GUI and callbacks happen here}

    \} \textcolor{keywordflow}{while} (key != \textcolor{charliteral}{'q'});

    destroyAllWindows();

    \textcolor{keywordflow}{return} 0;
\}
\end{DoxyCode}
 Notes\+:
\begin{DoxyItemize}
\item To expand the widget to the full extent of it\textquotesingle{}s parent layout use \hyperlink{classcanvascv_1_1Widget_a7b1ed6190950de22565c244f5aac49a4}{canvascv\+::\+Widget\+::set\+Stretch\+X\+To\+Parent()} with {\itshape true} to match the layout width and \hyperlink{classcanvascv_1_1Widget_a825028e2405bfdf9d92d372e60585703}{canvascv\+::\+Widget\+::set\+Stretch\+Y\+To\+Parent()} with {\itshape true} to match the layout height.
\item Here is a possible outcome of this code\+:  ~\newline

\end{DoxyItemize}\hypertarget{tutlayouts_layouts_s2}{}\subsection{A simple buttons group}\label{tutlayouts_layouts_s2}
Here we\textquotesingle{}re going to create some buttons in a group.

The buttons are going to be evenly spaced in a {\itshape Vertical\+Layout}.

The code is simple\+: 
\begin{DoxyCode}
\textcolor{preprocessor}{#include <canvascv/canvas.h>}
\textcolor{preprocessor}{#include <canvascv/widgets/button.h>}
\textcolor{preprocessor}{#include <canvascv/widgets/verticallayout.h>}

\textcolor{keyword}{using namespace }\hyperlink{namespacecv}{cv};
\textcolor{keyword}{using namespace }\hyperlink{namespacecanvascv}{canvascv};

\textcolor{keywordtype}{int} main(\textcolor{keywordtype}{int} argc, \textcolor{keywordtype}{char} **argv)
\{
    --argc;
    ++argv;
    \textcolor{keywordflow}{if} (! argc)
    \{
        Canvas::fatal(\textcolor{stringliteral}{"Must get a path to an image as a parameter"} , -1);
    \}

    Mat image = imread(argv[0]);
    \textcolor{keywordflow}{if} (image.empty())
    \{
        Canvas::fatal(\textcolor{keywordtype}{string}(\textcolor{stringliteral}{"Cannot load image "}) + argv[0], -2);
    \}

    namedWindow(\textcolor{stringliteral}{"LayoutButtons"}, WINDOW\_AUTOSIZE); \textcolor{comment}{// disable mouse resize}
    \hyperlink{classcanvascv_1_1Canvas}{Canvas} c(\textcolor{stringliteral}{"LayoutButtons"}, image.size());
    c.\hyperlink{classcanvascv_1_1Canvas_acf6e5d4b40aec610b0dc8c4f6bf93ac1}{setMouseCallback}();
    c.\hyperlink{classcanvascv_1_1Canvas_a402c43a42c0089c48a96e5303c1c1fe8}{enableStatusMsg}();

    \textcolor{keyword}{auto} vLayout = VerticalLayout::create(c, \{100,100\});

    Button::create(*vLayout,        \textcolor{comment}{// on the canvas}
                   \textcolor{stringliteral}{"First Button"},  \textcolor{comment}{// displayed on button}
                   \textcolor{stringliteral}{"one"});          \textcolor{comment}{// displayed during mouse hover}

    Button::create(*vLayout,        \textcolor{comment}{// on the canvas}
                   \textcolor{stringliteral}{"Second Button"}, \textcolor{comment}{// displayed on button}
                   \textcolor{stringliteral}{"two"});          \textcolor{comment}{// displayed during mouse hover}

    Button::create(*vLayout,        \textcolor{comment}{// on the canvas}
                   \textcolor{stringliteral}{"Third Button"},  \textcolor{comment}{// displayed on button}
                   \textcolor{stringliteral}{"three"});        \textcolor{comment}{// displayed during mouse hover}

    \textcolor{keywordtype}{int} key = 0;
    c.\hyperlink{classcanvascv_1_1Canvas_a441c5882c7ebebd454a306b3c3478ae7}{setImage}(image);
    \textcolor{keywordflow}{do}
    \{
        key = c.\hyperlink{classcanvascv_1_1Canvas_a59397db05f5d9e45264f626f6a2ae528}{waitKeyEx}(); \textcolor{comment}{// GUI and callbacks happen here}

    \} \textcolor{keywordflow}{while} (key != \textcolor{charliteral}{'q'});

    destroyAllWindows();

    \textcolor{keywordflow}{return} 0;
\}
\end{DoxyCode}
 Notes\+:
\begin{DoxyItemize}
\item When creating widgets the first parameter is the containing layout (the {\itshape Canvas} is also a {\itshape Layout}).
\item As you can see, there is a possible problem with this code -\/ the length of the buttons\+: 
\item To fix this, add these lines after the creation of the buttons\+: 
\begin{DoxyCode}
\textcolor{keyword}{using namespace }\hyperlink{namespacecanvascv}{canvascv};
\textcolor{comment}{// ...}

vLayout->at<\hyperlink{classcanvascv_1_1Button}{Button}>(0)->setStretchX(\textcolor{keyword}{true});
vLayout->at<\hyperlink{classcanvascv_1_1Button}{Button}>(1)->setStretchX(\textcolor{keyword}{true});
vLayout->at<\hyperlink{classcanvascv_1_1Button}{Button}>(2)->setStretchX(\textcolor{keyword}{true});

\textcolor{comment}{// ...}
\end{DoxyCode}
 Notes\+:
\item Notice how we can access the items in a layout by their index.
\item Again -\/ \hyperlink{classcanvascv_1_1Widget_a7cdddebd755c499712793f727a057733}{canvascv\+::\+Widget\+::set\+Stretch\+X()} and \hyperlink{classcanvascv_1_1Widget_a3ef50b76d33c332cea4e632346b6df33}{canvascv\+::\+Widget\+::set\+Stretch\+Y()} will stretch self to size of the largest widget in our layout.
\item Now the buttons have the same length\+:  ~\newline

\end{DoxyItemize}\hypertarget{tutlayouts_layouts_s3}{}\subsection{A full dialog}\label{tutlayouts_layouts_s3}
Now it\textquotesingle{}s time to take all the widgets we\textquotesingle{}ve been talking about in the tutorials and make a dialog frame for our user.

As always it\textquotesingle{}s going to be displayed on an image, but this time we\textquotesingle{}ll make sure the image size fits into the screen. Since Open\+CV doesn\textquotesingle{}t expose the desktop size, we\textquotesingle{}ll hard code a rough estimation of 1024x768 maximum size.


\begin{DoxyItemize}
\item The dialog will be in a {\itshape V\+Frame}.
\begin{DoxyItemize}
\item top -\/ title in a {\itshape R\+A\+I\+S\+ED} V\+Frame/\+H\+Frame
\item bottom -\/ body in a {\itshape H\+Frame} -\/ split into 2 horizontal parts\+:
\begin{DoxyItemize}
\item Left part will have a {\itshape S\+U\+N\+K\+EN} V\+Frame with
\begin{DoxyItemize}
\item {\itshape Radio\+Buttons}
\item {\itshape Check\+Boxes}
\end{DoxyItemize}
\item Right part will have a {\itshape V\+Frame} with
\begin{DoxyItemize}
\item {\itshape H\+Frame} with {\itshape Text} to display the user\textquotesingle{}s {\itshape Radio\+Buttons} and {\itshape Check\+Boxes} selections
\item {\itshape Horizontal\+Layout} with \char`\"{}\+Reset\char`\"{}/\char`\"{}\+Cancel\char`\"{}/\char`\"{}\+Ok\char`\"{} buttons (aligned to {\itshape B\+O\+T\+T\+OM})
\end{DoxyItemize}
\end{DoxyItemize}
\end{DoxyItemize}
\end{DoxyItemize}


\begin{DoxyCode}
\textcolor{preprocessor}{#include <canvascv/canvas.h>}
\textcolor{preprocessor}{#include <canvascv/widgets/button.h>}
\textcolor{preprocessor}{#include <canvascv/widgets/hframe.h>}
\textcolor{preprocessor}{#include <canvascv/widgets/vframe.h>}
\textcolor{preprocessor}{#include <canvascv/widgets/radiobuttons.h>}
\textcolor{preprocessor}{#include <canvascv/widgets/checkboxes.h>}
\textcolor{preprocessor}{#include <canvascv/widgets/text.h>}

\textcolor{keyword}{using namespace }\hyperlink{namespacecv}{cv};
\textcolor{keyword}{using namespace }\hyperlink{namespacecanvascv}{canvascv};

\textcolor{keywordtype}{void} buildDemoDialog(\hyperlink{classcanvascv_1_1Canvas}{Canvas} &c)
\{
    \textcolor{keyword}{auto} topFrame = VFrame::create(c, \{100,100\});
    \textcolor{keyword}{auto} titleFrame = VFrame::create(*topFrame);
    \textcolor{keyword}{auto} bodyFrame = HFrame::create(*topFrame);
    \textcolor{keyword}{auto} leftBodyFrame = VFrame::create(*bodyFrame);
    \textcolor{keyword}{auto} rightBodyFrame = VFrame::create(*bodyFrame);
    \textcolor{keyword}{auto} textInfoFrame = HFrame::create(*rightBodyFrame);
    \textcolor{keyword}{auto} buttonsLayout = HorizontalLayout::create(*rightBodyFrame);

    leftBodyFrame->setFrameRelief(Widget::SUNKEN);
    textInfoFrame->setFrameRelief(Widget::SUNKEN);

    \textcolor{comment}{// The title frame and body frame adapt to each other width}
    titleFrame->setStretchX(\textcolor{keyword}{true});
    bodyFrame->setStretchX(\textcolor{keyword}{true});

    rightBodyFrame->setStretchY(\textcolor{keyword}{true});

    \textcolor{comment}{// Adjust to titleFrame width and put the text at the center}
    \textcolor{keyword}{auto} title = Text::create(*titleFrame, \textcolor{stringliteral}{"Dialog title goes here"}, Widget::CENTER);
    title->setStretchXToParent(\textcolor{keyword}{true});
    title->setAlpha(0);
    titleFrame->setFrameRelief(Widget::RAISED);

    shared\_ptr<Text> textInfo = Text::create(*textInfoFrame, \textcolor{stringliteral}{""});
    shared\_ptr<RadioButtons> radioButtons = RadioButtons::create(*leftBodyFrame, \{\textcolor{stringliteral}{"RBOption1"}, \textcolor{stringliteral}{"RBOption2"}\}
      , 0);
    shared\_ptr<CheckBoxes> checkBoxes = CheckBoxes::create(*leftBodyFrame, \{\textcolor{stringliteral}{"CBOption1"}, \textcolor{stringliteral}{"CBOption2"}, \textcolor{stringliteral}{"
      CBOption3"}\});

    \textcolor{comment}{// The same callback will be used for several widgets}
    \hyperlink{classcanvascv_1_1Widget_a977cbd39cf203c5866f07f3645c7e4bc}{Widget::CBUserSelection} cb = [textInfo, radioButtons, checkBoxes](Widget*,int)
    \{
        stringstream s;
        \textcolor{keywordflow}{for} (\textcolor{keywordtype}{int} i = 0; i < checkBoxes->size(); ++i)
            s << \textcolor{stringliteral}{"CheckBox option '"} << checkBoxes->\hyperlink{classcanvascv_1_1CheckBoxes_a1ca004ddd840090415924b1f79b2ee47}{getTextAt}(i) << \textcolor{stringliteral}{"' is "} << checkBoxes->
      isChecked(i) << \textcolor{stringliteral}{"\(\backslash\)n"};
        s << \textcolor{stringliteral}{"RadioButtons selection is '"} << radioButtons->getTextAt(radioButtons->getSelection())<< \textcolor{stringliteral}{"'\(\backslash\)n"}
      ;
        textInfo->setText(s.str());
    \};

    cb(0,0); \textcolor{comment}{// manually initialize the textInfo by invoking the callback}
    textInfo->setAlpha(0);

    radioButtons->setUserCB(cb);
    checkBoxes->setUserCB(cb);

    Button::create(*buttonsLayout, \textcolor{stringliteral}{"Reset"})->onPress([radioButtons, checkBoxes](Widget *)
    \{
        \textcolor{keywordflow}{for} (\textcolor{keywordtype}{int} i = 0; i < checkBoxes->size(); ++i)
            checkBoxes->setChecked(i, \textcolor{keyword}{false});
        radioButtons->setSelection(0);
    \});
    Button::create(*buttonsLayout, \textcolor{stringliteral}{"Cancel"});
    Button::create(*buttonsLayout, \textcolor{stringliteral}{"Ok"});
    buttonsLayout->setStretchYToParent(\textcolor{keyword}{true});
    buttonsLayout->setLayoutAnchor(Widget::CENTER);
    buttonsLayout->doForAll([](Widget *w)
    \{
        w->\hyperlink{classcanvascv_1_1Widget_a7cdddebd755c499712793f727a057733}{setStretchX}(\textcolor{keyword}{true});
        w->\hyperlink{classcanvascv_1_1Widget_a7180ca0054874c853e805e7842741596}{setLayoutAnchor}(Widget::BOTTOM);
    \}, 1, \textcolor{keyword}{false});
\}

\textcolor{keywordtype}{int} main(\textcolor{keywordtype}{int} argc, \textcolor{keywordtype}{char} **argv)
\{
    --argc;
    ++argv;
    Mat image;
    \textcolor{keywordflow}{if} (argc)
    \{
        Mat orig = imread(argv[0]);
        \textcolor{keywordflow}{if} (orig.empty())
        \{
            Canvas::fatal(\textcolor{keywordtype}{string}(\textcolor{stringliteral}{"Cannot load image "}) + argv[0], -1);
        \}
        \textcolor{keywordflow}{if} (orig.cols > 1024)
        \{
            \textcolor{keywordtype}{double} ratio = 1024. / orig.cols;
            cv::resize(orig, image, Size(), ratio, ratio);
        \}
        \textcolor{keywordflow}{else}
        \{
            image = orig;
        \}
    \}
    \textcolor{keywordflow}{else}
    \{
        Canvas::fatal(\textcolor{stringliteral}{"Must get a path to an image as a parameter"} , -2);
    \}

    namedWindow(\textcolor{stringliteral}{"Dialog"}, WINDOW\_AUTOSIZE); \textcolor{comment}{// disable mouse resize}
    moveWindow(\textcolor{stringliteral}{"Dialog"}, 10, 10); \textcolor{comment}{// optional - some control on window position}
    \hyperlink{classcanvascv_1_1Canvas}{Canvas} c(\textcolor{stringliteral}{"Dialog"}, image.size());
    c.\hyperlink{classcanvascv_1_1Canvas_acf6e5d4b40aec610b0dc8c4f6bf93ac1}{setMouseCallback}();
    c.\hyperlink{classcanvascv_1_1Canvas_a402c43a42c0089c48a96e5303c1c1fe8}{enableStatusMsg}();

    buildDemoDialog(c);

    \textcolor{keywordtype}{int} key = 0;
    c.\hyperlink{classcanvascv_1_1Canvas_a441c5882c7ebebd454a306b3c3478ae7}{setImage}(image);
    \textcolor{keywordflow}{do}
    \{
        key = c.\hyperlink{classcanvascv_1_1Canvas_a59397db05f5d9e45264f626f6a2ae528}{waitKeyEx}(); \textcolor{comment}{// GUI and callbacks happen here}

    \} \textcolor{keywordflow}{while} (key != \textcolor{charliteral}{'q'});

    destroyAllWindows();

    \textcolor{keywordflow}{return} 0;
\}
\end{DoxyCode}
 Notes\+:
\begin{DoxyItemize}
\item Creating the frames and layouts first is recommended. Everything is inserted into them after that.
\item We\textquotesingle{}re using the same callback here for handling changes in the {\itshape Check\+Boxes} and {\itshape Radio\+Buttons}.
\item Since widgets are layered on top of each other, you might want to make some of the widget transparent, by using \hyperlink{classcanvascv_1_1Widget_a4c16525b31e70acd43372af0c1e60d42}{canvascv\+::\+Widget\+::set\+Alpha()}.
\item You can perform actions recursively on {\itshape Compound\+Widgets} with \hyperlink{classcanvascv_1_1CompoundWidget_aa7d0f488468fca5707aa49ea35e9c67e}{canvascv\+::\+Compound\+Widget\+::do\+For\+All()}.
\item This is what you should get\+:  ~\newline

\end{DoxyItemize}

{\bfseries That\textquotesingle{}s all for this tutorial} 